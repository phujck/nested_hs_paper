\section{\label{sec:commuting}Commuting Sector Theorem}

We now come to the centrepiece of the paper. Suppose the system coupling operator commutes with the bare Hamiltonian:
\begin{equation}
    [H_S,\,f] = 0.
    \label{eq:commuting_assumption}
\end{equation}
This is the ``quantum non-demolition'' or pure-dephasing sector: the bath monitors a quantity that is conserved under the free system dynamics. Physically, this arises whenever the environment couples to a constant of motion---total charge, magnetisation, particle number, or any operator in the centre of the system algebra. It also describes the important class of longitudinal noise channels in qubit systems.

The algebraic consequence is immediate and dramatic. Since $[H_S,f]=0$, the interaction-picture coupling operator loses its $\tau$-dependence:
\begin{equation}
    f(\tau) = e^{\tau H_S}\,f\,e^{-\tau H_S} = f.
    \label{eq:f_tau_independent}
\end{equation}
Time ordering becomes redundant: $\mathcal{T}_\tau\,\exp[\cdots] = \exp[\cdots]$,  and all the $n$-local composites collapse:
\begin{equation}
    \Psi_n(\tau_1,\ldots,\tau_n) = \prod_{j=1}^n f(\tau_j) = f^n.
    \label{eq:Psin_commuting}
\end{equation}

\subsection{Collapse of the auxiliary fields}

Consider the $n$-th level of the nested HS hierarchy from Eq.~\eqref{eq:nested_hs_theorem}. In the commuting sector, the exponent becomes
\begin{widetext}
\begin{equation}
    \frac{1}{n!}\int\!\eta_n(\tau_1,\ldots,\tau_n)\;\Psi_n\;d\tau_1\cdots d\tau_n
    =
    \frac{f^n}{n!}\underbrace{\int\!\eta_n(\tau_1,\ldots,\tau_n)\;d\tau_1\cdots d\tau_n}_{=:\,X_n}.
    \label{eq:Xn_def}
\end{equation}
\end{widetext}
All the multilocal structure has collapsed: the $n$-point auxiliary field $\eta_n$ contributes only through its total ``charge'' $X_n$. Since $\eta_n$ is Gaussian, $X_n$ is a scalar Gaussian variable.

The variance of $X_n$ is obtained by integrating the covariance of $\eta_n$ over all $2n$ time arguments:
\begin{widetext}
\begin{align}
    \mathbb{E}[X_n^2]
    &=
    \int\!\mathbb{E}[\eta_n(\tau_1,\ldots,\tau_n)\,\eta_n(\tau_1',\ldots,\tau_n')]\;d\tau_1\cdots d\tau_n\,d\tau_1'\cdots d\tau_n'
    =
    \frac{2(n!)^2}{(2n)!}\int\! K_{2n}^{(n|n)}\;d\tau_1\cdots d\tau_{2n}.
    \label{eq:Xn_variance}
\end{align}
\end{widetext}
But in the commuting sector, the projection $K_{2n}^{(n|n)}$ is redundant---since all time arguments enter symmetrically after integration, every pairing contributes equally:
\begin{equation}
    \int K_{2n}^{(n|n)} = \int K_{2n}.
    \label{eq:projection_trivial}
\end{equation}
Therefore
\begin{equation}
    \mathbb{E}[X_n^2] = \frac{2(n!)^2}{(2n)!}\int_0^\beta d\tau_1\cdots d\tau_{2n}\;K_{2n}(\tau_1,\ldots,\tau_{2n}).
    \label{eq:Xn_var_explicit}
\end{equation}

\subsection{Analytic re-averaging}

The $n$-th level HS average now reads
\begin{equation}
    \mathbb{E}_{\eta_n}\!\left[e^{f^n X_n/n!}\right]
    =
    \exp\!\left[\frac{f^{2n}}{2(n!)^2}\,\mathbb{E}[X_n^2]\right],
    \label{eq:gaussian_moment}
\end{equation}
since the moment generating function of a zero-mean Gaussian $X$ with variance $\sigma^2$ is $\mathbb{E}[e^{tX}]=e^{t^2\sigma^2/2}$. Substituting Eq.~\eqref{eq:Xn_var_explicit}:
\begin{align}
    \mathbb{E}_{\eta_n}\!\left[e^{f^n X_n/n!}\right]
    &=
    \exp\!\left[\frac{f^{2n}}{2(n!)^2}\cdot\frac{2(n!)^2}{(2n)!}\int K_{2n}\right]
    \nonumber\\
    &=
    \exp\!\left[\frac{f^{2n}}{(2n)!}\int K_{2n}\right]
    \nonumber\\
    &=
    \exp\!\left[\alpha_{2n}\,f^{2n}\right],
    \label{eq:level_n_result}
\end{align}
where we define the integrated cumulant coefficient
\begin{equation}
    \alpha_{2n} := \frac{1}{(2n)!}\int_0^\beta d\tau_1\cdots d\tau_{2n}\;K_{2n}(\tau_1,\ldots,\tau_{2n}).
    \label{eq:alpha2n_def}
\end{equation}
Everything cancels beautifully: the $(n!)^2$ from the HS covariance formula eats the $(n!)^2$ in the denominator from the coupling $f^n/n!$ squared, and what remains is the natural cumulant prefactor $1/(2n)!$. 

\subsection{The theorem}

Combining all levels of the hierarchy, the reduced equilibrium operator becomes

\medskip
\noindent\fbox{\parbox{0.95\columnwidth}{%
\textbf{Commuting Sector Theorem.} \emph{If\/ $[H_S,f]=0$ and all odd cumulants vanish, then}
\begin{equation}
    \bar\rho_S(\beta)
    =
    Z_B\,\exp\!\left(-\beta H_S + \sum_{n=1}^{\infty}\alpha_{2n}\,f^{2n}\right),
    \label{eq:commuting_theorem}
\end{equation}
\emph{with $\alpha_{2n}$ given by Eq.~\eqref{eq:alpha2n_def}. Equivalently, the Hamiltonian of mean force is}
\begin{equation}
    H_{\mathrm{MF}}(\beta) = H_S - \frac{1}{\beta}\sum_{n=1}^{\infty}\alpha_{2n}(\beta)\,f^{2n}.
    \label{eq:HMF_commuting_full}
\end{equation}
}}
\medskip

\noindent Several points are worth emphasising:

\begin{enumerate}
    \item \textbf{Exactness.} This is not a perturbative result. It holds for arbitrary coupling strength, arbitrary temperature, and arbitrary (even non-Gaussian) bath statistics. No truncation of the cumulant series is performed; the theorem holds order by order, and resumming is exact.

    \item \textbf{No PSD requirement.} Nowhere did we assume that any cumulant kernel $K_{2n}$ is positive semi-definite. The HS transformation is valid for indefinite and complex kernels, with convergence guaranteed by contour rotation. In physical terms: baths with sub-Gaussian tails (like bounded spin baths) generically produce $K_4 < 0$, and the construction handles this without difficulty.

    \item \textbf{Contour rotation suffices.} For complex $\alpha_{2n}$ (arising from complex integrated cumulants), the exponential in Eq.~\eqref{eq:commuting_theorem} is well-defined as a matrix function in any eigenbasis of $f$. Hermiticity of $\bar\rho_S$ is ensured by the Hermiticity constraints on the cumulants of a self-adjoint operator $B$.

    \item \textbf{Gaussian limit.} Setting $K_n = 0$ for $n \geq 3$ gives $\alpha_2 = \frac{1}{2}\int\!\!\int K_2(\tau,\tau')\,d\tau\,d\tau' = \frac{1}{2}C(\beta)$, reproducing $-\lambda f^2$ with $\lambda = C(\beta)/(2\beta)$---the reorganisation energy result of the companion paper.
\end{enumerate}

The theorem tells us that the entire bath influence, in the commuting sector, enters as a renormalised polynomial in $f$. The coefficients $\alpha_{2n}$ are nothing but the integrated cumulants of the bath coupling operator, weighted by the natural combinatorial factor $1/(2n)!$. Each cumulant order contributes one new power of $f^2$ to the Hamiltonian of mean force.
