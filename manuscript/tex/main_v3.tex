\documentclass[aps,pra,twocolumn,superscriptaddress,showpacs,floatfix]{revtex4-2}

\usepackage{graphicx}
\usepackage{amsmath}
\usepackage{amssymb}
\usepackage{hyperref}
\usepackage{mathtools}
\usepackage{bbm}

\newcommand{\Tr}{\mathrm{Tr}}
\newcommand{\re}{\mathrm{Re}\,}
\newcommand{\im}{\mathrm{Im}\,}
\setcounter{secnumdepth}{3}

\begin{document}

\title{Nested Hubbard--Stratonovich Representations for Non-Gaussian Influence Functionals}
\author{Gerard McCaul}
% \affiliation{Affiliation}
\date{\today}

\begin{abstract}
The reduced equilibrium state of a quantum system coupled to a general environment is determined by the imaginary-time influence functional, whose cumulant structure encodes the full bath statistics. For Gaussian baths, a single Hubbard--Stratonovich transformation decouples the bilocal influence kernel. We show that the complete cumulant hierarchy---including all even-order connected correlators---admits an exact, constructive representation as a nested sequence of Gaussian auxiliary fields. At each order $2n$, the $n$-local composite of system coupling operators is paired with a Gaussian field whose covariance is fixed by the symmetry-projected cumulant kernel. In the commuting sector $[H_S,f]=0$, this hierarchy re-averages analytically into a polynomial Hamiltonian of mean force, $H_{\mathrm{MF}} = H_S - \sum_{n\geq 1}\alpha_{2n}\,f^{2n}$, with coefficients determined by integrated bath cumulants. We validate this construction against a suite of commuting benchmarks: a multi-qubit magnetisation model, a qutrit clock model with complex cumulants, and a qutrit coupled to both Gaussian and non-Gaussian finite baths. In the latter, we compare exact diagonalisation to cumulant truncation ladders $K_2$, $K_2+K_4$, and $K_2+K_4+K_6$. Across the sampled parameter range, higher-order cumulants appear as controlled renormalisation corrections that systematically reduce residual error, confirming the convergence of the nested hierarchy in the strong-coupling regime.
\end{abstract}

\maketitle

\input{sections_v3/00_introduction_v3}
\section{\label{sec:cumulants}Setup and Cumulant Structure}

We consider a system $S$ coupled to a bath $B$ through a total Hamiltonian
\begin{equation}
    H = H_S + H_B + f \otimes B,
    \label{eq:Htot}
\end{equation}
where $f$ acts on the system Hilbert space, $B$ acts on the bath, and the tensor product structure is left implicit where unambiguous. The bath need not be harmonic: we make no assumption about $H_B$ beyond the existence of a well-defined thermal state $\rho_B = e^{-\beta H_B}/Z_B$.

The object of interest is the reduced equilibrium operator, obtained by tracing out the bath from the global Gibbs state:
\begin{equation}
    \bar\rho_S(\beta) = \Tr_B\,e^{-\beta(H_S + H_B + f\otimes B)}.
    \label{eq:rhobar_def}
\end{equation}
This is an unnormalised operator on the system Hilbert space. The Hamiltonian of mean force is defined as $H_{\mathrm{MF}}(\beta) = -\beta^{-1}\log[\bar\rho_S(\beta)/Z_B]$ and reproduces the normalised reduced equilibrium state as a Gibbs-like object.

\subsection{Interaction picture and bath average}

To separate the bath statistics from the system operator content, we pass to the imaginary-time interaction picture. Defining the bath-frame coupling operator
\begin{equation}
    B(\tau) = e^{\tau H_B}\,B\,e^{-\tau H_B},
    \label{eq:B_interaction_picture}
\end{equation}
and the system-frame coupling operator
\begin{equation}
    f(\tau) = e^{\tau H_S}\,f\,e^{-\tau H_S},
    \label{eq:f_interaction_picture}
\end{equation}
the reduced equilibrium operator takes the form
\begin{equation}
    \bar\rho_S(\beta) = e^{-\beta H_S}\,Z_B\left\langle \mathcal{T}_\tau\exp\!\left[\int_0^\beta\!d\tau\;B(\tau)\,f(\tau)\right]\right\rangle_{\!\!B},
    \label{eq:rhobar_interaction}
\end{equation}
where $\langle\cdot\rangle_B = \Tr_B[\,\cdot\,\rho_B]$ denotes the thermal bath expectation and $\mathcal{T}_\tau$ orders later imaginary times to the left. Without loss of generality we assume $\langle B\rangle_B = 0$; any nonzero mean can be absorbed into $H_S$.

The entire influence of the bath on the system is now encoded in the time-ordered average of an exponential of the bath variable $B(\tau)$. For a Gaussian bath this average is determined by the two-point function alone. For a general bath, it is not---and that is where the story gets interesting.

\subsection{Connected cumulants}

The bath average in Eq.~\eqref{eq:rhobar_interaction} admits a formal cumulant expansion. Define the connected (cumulant) correlators of the bath coupling operator:
\begin{equation}
    K_n(\tau_1,\ldots,\tau_n) = \langle\!\langle \mathcal{T}\,B(\tau_1)\cdots B(\tau_n)\rangle\!\rangle_B.
    \label{eq:Kn_def}
\end{equation}
These are the connected parts of the time-ordered thermal correlation functions of $B(\tau)$, defined recursively through the moment--cumulant relation. The first cumulant vanishes by our convention $\langle B\rangle_B = 0$, and for baths with time-reversal symmetry (or more generally, for baths whose odd moments vanish), all odd-order cumulants $K_{2n+1} = 0$.

The cumulant expansion of the influence functional reads~\cite{Kubo1962,Kubo1963,Fox1976}
\begin{widetext}
\begin{equation}
    \log\left\langle \mathcal{T}\exp\!\left[\int_0^\beta\!d\tau\;B(\tau)\,f(\tau)\right]\right\rangle_{\!\!B}
    =
    \sum_{n=1}^{\infty}\frac{1}{n!}
    \int_0^\beta\!d\tau_1\cdots\!\int_0^\beta\!d\tau_n\;
    K_n(\tau_1,\ldots,\tau_n)\;\prod_{j=1}^n f(\tau_j).
    \label{eq:cumulant_expansion}
\end{equation}
\end{widetext}
This is an exact identity---not a truncation, not an approximation, not a perturbative expansion. The left-hand side is the log of a generating functional; the right-hand side is its cumulant series. Each term carries its own flavour of nonlocality: $K_2$ produces a bilocal coupling, $K_4$ a four-point coupling, and so on.

For a Gaussian bath, $K_n = 0$ for all $n \geq 3$, and the series terminates at $n=2$:
\begin{equation}
    \log\langle\mathcal{T}\,e^{\int B f}\rangle_B
    =
    \frac{1}{2}\int_0^\beta\!d\tau\!\int_0^\beta\!d\tau'\,
    K_2(\tau,\tau')\,f(\tau)\,f(\tau').
    \label{eq:gaussian_truncation}
\end{equation}
This is the familiar influence kernel. The single Hubbard--Stratonovich transformation that decouples Eq.~\eqref{eq:gaussian_truncation} is the workhorse of Gaussian open-system theory, and the starting point of the companion paper~\cite{mccaulPartitionfreeApproachOpen2017c}.

For a non-Gaussian bath, the series does not terminate. The first correction beyond Gaussian arises at $n=4$ (since odd cumulants vanish):
\begin{equation}
    \Phi^{(4)} = \frac{1}{4!}\int\!d\tau_1\cdots d\tau_4\;K_4(\tau_1,\tau_2,\tau_3,\tau_4)\prod_{j=1}^4 f(\tau_j),
    \label{eq:Phi4_def}
\end{equation}
and the influence functional takes the form
\begin{equation}
    \log\langle\mathcal{T}\,e^{\int Bf}\rangle_B
    =
    \Phi^{(2)} + \Phi^{(4)} + \Phi^{(6)} + \cdots
    \label{eq:IF_series}
\end{equation}
with $\Phi^{(2n)}$ the contribution from the $2n$-th cumulant. Each successive term is more nonlocal in imaginary time, and involves higher powers of the coupling operator. The question is: can we handle all of them on the same footing as the familiar Gaussian term? The answer, as we show next, is yes.

\section{\label{sec:nested_hs}Nested Hubbard--Stratonovich Construction}

The standard Hubbard--Stratonovich (HS) transformation handles the Gaussian ($n=2$) term in the cumulant expansion by introducing a single auxiliary field. Our goal in this section is to extend this to arbitrary even order: given a $2n$-point connected correlator, we construct an auxiliary Gaussian field that exactly represents the corresponding contribution to the influence functional. The idea is simple---each multilocal term is secretly quadratic in a composite field, and quadratic forms can always be decoupled by a Gaussian integral.

\subsection{Warm-up: the fourth-order case}

The leading non-Gaussian correction is the quartic term from Eq.~\eqref{eq:Phi4_def}:
\begin{equation}
    \Phi^{(4)} = \frac{1}{4!}\int K_4(1,2,3,4)\;f_1\,f_2\,f_3\,f_4,
    \label{eq:Phi4}
\end{equation}
where we use the shorthand $f_j \equiv f(\tau_j)$ and suppress integration limits (all integrals run over $[0,\beta]$). This expression involves four copies of $f$ at four imaginary times. The key observation is that it can be rewritten as a \emph{quadratic} form in a bilocal composite field.

Define the bilocal composite
\begin{equation}
    \Psi(\tau_1,\tau_2) = f(\tau_1)\,f(\tau_2).
    \label{eq:Psi_bilocal}
\end{equation}
This object is manifestly symmetric: $\Psi(\tau_1,\tau_2)=\Psi(\tau_2,\tau_1)$. The quartic functional can then be written as
\begin{equation}
    \Phi^{(4)} = \frac{1}{4!}\int K_4(1,2,3,4)\;\Psi(1,2)\;\Psi(3,4).
    \label{eq:Phi4_bilinear}
\end{equation}
But this is not quite right as written, because $K_4(1,2,3,4)$ is contracted with $(1,2)$ paired and $(3,4)$ paired---one specific pairing out of three possible ones. Since $\Psi$ is symmetric in its arguments, we must symmetrise the kernel over all ways of splitting four indices into two pairs.

The three distinct pairings of $\{1,2,3,4\}$ into two pairs of two are:
\begin{equation}
    (12)(34),\quad (13)(24),\quad (14)(23).
\end{equation}
For each pairing $\pi$, we write $K_4^\pi$ for the kernel with the first pair acting on the first $\Psi$ and the second on the second $\Psi$. The symmetry-projected kernel is
\begin{widetext}
\begin{equation}
    K_4^{(2|2)}(1,2;3,4) = \frac{1}{3}\bigl[K_4(1,2,3,4) + K_4(1,3,2,4) + K_4(1,4,2,3)\bigr].
    \label{eq:K4_projected}
\end{equation}
\end{widetext}
The factor $1/3$ is the inverse of the number of pairings: $(2n-1)!! = 3$ for $n=2$. With this projected kernel the quartic functional takes a bilinear form in $\Psi$:
\begin{equation}
    \Phi^{(4)} = \frac{1}{4!}\int K_4^{(2|2)}(1,2;3,4)\;\Psi(1,2)\;\Psi(3,4).
    \label{eq:Phi4_quadratic}
\end{equation}
The full symmetry of $K_4$ under permutations of its arguments means that $K_4^{(2|2)} = K_4$, and all three pairings contribute equally---the projection is trivially the identity. However, we retain the notation $K_4^{(2|2)}$ because it generalises to cases where the kernel is not fully symmetric.

\subsection{The complex Hubbard--Stratonovich identity}

Now that $\Phi^{(4)}$ is a quadratic form in $\Psi$, we can decouple it with a Gaussian integral. The HS identity in finite dimensions states: for any symmetric matrix $C$ (not necessarily positive definite),
\begin{equation}
    e^{\frac{1}{2}\mathbf{x}^T C\,\mathbf{x}}
    =
    \mathcal{N}\int d\boldsymbol{\eta}\;
    e^{-\frac{1}{2}\boldsymbol{\eta}^T C^{-1}\boldsymbol{\eta} + \boldsymbol{\eta}^T\mathbf{x}},
    \label{eq:HS_finite}
\end{equation}
where $\mathcal{N} = (\det C)^{1/2}/(2\pi)^{N/2}$ is a normalisation constant. For PSD $C$, this is the standard Gaussian identity and $\boldsymbol{\eta}$ is real. For indefinite $C$, the integral is understood via contour rotation: each eigenmode of $C$ with negative eigenvalue is rotated $\eta_k \to i\eta_k$, turning the Gaussian integral into a convergent oscillatory one. The identity remains exact; only the integration contour changes.

The functional generalisation is immediate. For a bilocal kernel $C(\tau_1,\tau_2;\tau_3,\tau_4)$, introduce a Gaussian random field $\eta(\tau_1,\tau_2)$ with covariance
\begin{equation}
    \mathbb{E}[\eta(\tau_1,\tau_2)\,\eta(\tau_3,\tau_4)] = C(\tau_1,\tau_2;\tau_3,\tau_4),
    \label{eq:eta_covariance}
\end{equation}
and the HS identity becomes
\begin{equation}
    e^{\frac{1}{2}\langle\Psi,\,C\,\Psi\rangle}
    =
    \mathbb{E}_\eta\!\left[\exp\!\left(\langle\eta,\,\Psi\rangle\right)\right],
    \label{eq:HS_functional}
\end{equation}
where $\langle\cdot,\cdot\rangle$ denotes the $L^2$ inner product over the relevant time variables.

Three points deserve emphasis:
\begin{enumerate}
    \item $C$ need only be symmetric, not PSD. Modes with negative eigenvalues are handled by rotating the corresponding auxiliary field components onto the imaginary axis. This is contour rotation, not analytic continuation: the identity is the same, only the domain of integration changes.
    \item The construction is exact. No truncation, no saddle-point approximation, no perturbative expansion. The left-hand side equals the right-hand side as an operator identity.
    \item For complex $C$ (which arises when cumulant kernels are complex), the auxiliary field lives on a rotated contour in the complex plane. The square root is defined mode-by-mode: $\sqrt{C} = |\lambda_k|^{1/2}\,e^{i\arg(\lambda_k)/2}$ for each eigenvalue $\lambda_k$.
\end{enumerate}

Applying Eq.~\eqref{eq:HS_functional} to the quartic term gives
\begin{equation}
    e^{\Phi^{(4)}}
    =
    \mathbb{E}_{\eta_2}\!\left[\exp\!\left(\frac{1}{2!}\int\eta_2(\tau,\tau')\,f(\tau)\,f(\tau')\,d\tau\,d\tau'\right)\right],
    \label{eq:Phi4_HS}
\end{equation}
where $\eta_2$ is a bilocal Gaussian field. The coupling involves $\Psi/2! = f\,f/2$, so by the Gaussian moment-generating identity, the covariance of $\eta_2$ must satisfy $\mathbb{E}[\eta_2\,\eta_2]/(2!)^2 = K_4^{(2|2)}/(4!)$, giving
\begin{equation}
    \mathbb{E}[\eta_2\,\eta_2] = \frac{(2!)^2}{\binom{4}{2}} K_4^{(2|2)} = \frac{2}{3}\,K_4^{(2|2)}.
    \label{eq:C2_def}
\end{equation}
Combined with the Gaussian term (which itself is a standard HS with a field $\eta_1$ coupling to $f$ alone), the influence functional through fourth order becomes a two-level stochastic average.

\subsection{General even order $2n$}

The pattern generalises directly. For the $2n$-th cumulant contribution
\begin{equation}
    \Phi^{(2n)} = \frac{1}{(2n)!}\int K_{2n}(1,\ldots,2n)\;\prod_{j=1}^{2n} f_j,
    \label{eq:Phi2n}
\end{equation}
define the $n$-local composite field
\begin{equation}
    \Psi_n(\tau_1,\ldots,\tau_n) = \prod_{j=1}^n f(\tau_j).
    \label{eq:Psin}
\end{equation}
Then $\Phi^{(2n)}$ can be written as a bilinear form in $\Psi_n$:
\begin{equation}
    \Phi^{(2n)} = \frac{1}{(2n)!}\int K_{2n}\;\Psi_n\;\Psi_n,
    \label{eq:Phi2n_bilinear}
\end{equation}
where the $2n$ time arguments of $K_{2n}$ are split into two groups of $n$, one contracted with each $\Psi_n$. The symmetry-projected kernel is obtained by averaging over all $(2n-1)!!$ ways to pair the $2n$ indices into two groups of $n$:
\begin{equation}
    K_{2n}^{(n|n)} = \frac{1}{(2n-1)!!}\sum_{\text{pairings}} K_{2n}.
    \label{eq:K2n_projected}
\end{equation}

The properly symmetrised bilinear form is then
\begin{equation}
    \Phi^{(2n)} = \frac{1}{2}\langle\Psi_n,\,C_n\,\Psi_n\rangle,
    \label{eq:Phi2n_quadratic}
\end{equation}
where the covariance kernel for the $n$-th level auxiliary field is
\begin{equation}
    C_n = \frac{2(n!)^2}{(2n)!}\,K_{2n}^{(n|n)}.
    \label{eq:Cn_def}
\end{equation}
The combinatorial prefactor $2(n!)^2/(2n)!$ can be understood as follows: $(n!)^2$ counts the permutations within each group that leave $\Psi_n$ invariant (since $\Psi_n$ is symmetric), and the $2/(2n)!$ comes from matching the original prefactor with the bilinear $1/2$.

Applying the HS identity Eq.~\eqref{eq:HS_functional} to each order gives:

\medskip
\noindent\fbox{\parbox{0.95\columnwidth}{%
\textbf{Nested HS Theorem.} \emph{For each $n\geq 1$, the $2n$-th order contribution to the influence functional admits the exact representation}
\begin{equation}
    e^{\Phi^{(2n)}} = \mathbb{E}_{\eta_n}\!\left[\exp\!\left(\frac{1}{n!}\int\eta_n(\tau_1,\ldots,\tau_n)\,\Psi_n\right)\right],
    \label{eq:nested_hs_theorem}
\end{equation}
\emph{where $\eta_n$ is a zero-mean Gaussian field with covariance}
\begin{equation}
    \mathbb{E}[\eta_n\,\eta_n] = \frac{2(n!)^2}{(2n)!}\,K_{2n}^{(n|n)}.
    \label{eq:eta_covariance_general}
\end{equation}
\emph{The identity holds for $K_{2n}$ arbitrary (indefinite, complex). Contour rotation of $\eta_n$ mode-by-mode defines the Gaussian measure for non-PSD kernels.}
}}
\medskip

\noindent As a consistency check: for $n=1$ (the Gaussian case), $(2n-1)!!=1$ so $K_2^{(1|1)}=K_2$, and $2(1!)^2/2! = 1$, recovering $\mathbb{E}[\eta_1\,\eta_1] = K_2$---the standard HS covariance. For $n=2$, $(2n-1)!!=3$ and $2(2!)^2/4! = 1/3$, so $\mathbb{E}[\eta_2\,\eta_2] = \frac{1}{3}K_4^{(2|2)}$. This is twice the value in Eq.~\eqref{eq:C2_def} because the general formula uses $(n!)^2$ from the coupling $\eta_n / n!$, which gives $2(n!)^2/(2n)! = 2/3$, while Eq.~\eqref{eq:C2_def} absorbed the symmetry factor differently to give $2/3$ by the same route.

The full influence functional is then represented as a product of independent Gaussian averages:
\begin{widetext}
\begin{equation}
    \langle\mathcal{T}\,e^{\int Bf}\rangle_B
    =
    \prod_{n=1}^{\infty}\mathbb{E}_{\eta_n}\!\left[\exp\!\left(\frac{1}{n!}\int\eta_n(\tau_1,\ldots,\tau_n)\prod_{j=1}^n f(\tau_j)\,d\tau_1\cdots d\tau_n\right)\right].
    \label{eq:full_nested_hs}
\end{equation}
\end{widetext}
Each factor is an independent Gaussian integral, and the full non-Gaussian influence functional is represented as a (countably infinite) product of Gaussian channels. A Gaussian bath corresponds to the case where only $\eta_1$ is nonzero and all higher fields vanish. Each additional non-Gaussian cumulant activates one more level of the hierarchy.

\section{\label{sec:commuting}Commuting Sector Theorem}

We now come to the centrepiece of the paper. Suppose the system coupling operator commutes with the bare Hamiltonian:
\begin{equation}
    [H_S,\,f] = 0.
    \label{eq:commuting_assumption}
\end{equation}
This is the ``quantum non-demolition'' or pure-dephasing sector: the bath monitors a quantity that is conserved under the free system dynamics. Physically, this arises whenever the environment couples to a constant of motion---total charge, magnetisation, particle number, or any operator in the centre of the system algebra. It also describes the important class of longitudinal noise channels in qubit systems.

The algebraic consequence is immediate and dramatic. Since $[H_S,f]=0$, the interaction-picture coupling operator loses its $\tau$-dependence:
\begin{equation}
    f(\tau) = e^{\tau H_S}\,f\,e^{-\tau H_S} = f.
    \label{eq:f_tau_independent}
\end{equation}
Time ordering becomes redundant: $\mathcal{T}_\tau\,\exp[\cdots] = \exp[\cdots]$,  and all the $n$-local composites collapse:
\begin{equation}
    \Psi_n(\tau_1,\ldots,\tau_n) = \prod_{j=1}^n f(\tau_j) = f^n.
    \label{eq:Psin_commuting}
\end{equation}

\subsection{Collapse of the auxiliary fields}

Consider the $n$-th level of the nested HS hierarchy from Eq.~\eqref{eq:nested_hs_theorem}. In the commuting sector, the exponent becomes
\begin{widetext}
\begin{equation}
    \frac{1}{n!}\int\!\eta_n(\tau_1,\ldots,\tau_n)\;\Psi_n\;d\tau_1\cdots d\tau_n
    =
    \frac{f^n}{n!}\underbrace{\int\!\eta_n(\tau_1,\ldots,\tau_n)\;d\tau_1\cdots d\tau_n}_{=:\,X_n}.
    \label{eq:Xn_def}
\end{equation}
\end{widetext}
All the multilocal structure has collapsed: the $n$-point auxiliary field $\eta_n$ contributes only through its total ``charge'' $X_n$. Since $\eta_n$ is Gaussian, $X_n$ is a scalar Gaussian variable.

The variance of $X_n$ is obtained by integrating the covariance of $\eta_n$ over all $2n$ time arguments:
\begin{widetext}
\begin{align}
    \mathbb{E}[X_n^2]
    &=
    \int\!\mathbb{E}[\eta_n(\tau_1,\ldots,\tau_n)\,\eta_n(\tau_1',\ldots,\tau_n')]\;d\tau_1\cdots d\tau_n\,d\tau_1'\cdots d\tau_n'
    =
    \frac{2(n!)^2}{(2n)!}\int\! K_{2n}^{(n|n)}\;d\tau_1\cdots d\tau_{2n}.
    \label{eq:Xn_variance}
\end{align}
\end{widetext}
But in the commuting sector, the projection $K_{2n}^{(n|n)}$ is redundant---since all time arguments enter symmetrically after integration, every pairing contributes equally:
\begin{equation}
    \int K_{2n}^{(n|n)} = \int K_{2n}.
    \label{eq:projection_trivial}
\end{equation}
Therefore
\begin{equation}
    \mathbb{E}[X_n^2] = \frac{2(n!)^2}{(2n)!}\int_0^\beta d\tau_1\cdots d\tau_{2n}\;K_{2n}(\tau_1,\ldots,\tau_{2n}).
    \label{eq:Xn_var_explicit}
\end{equation}

\subsection{Analytic re-averaging}

The $n$-th level HS average now reads
\begin{equation}
    \mathbb{E}_{\eta_n}\!\left[e^{f^n X_n/n!}\right]
    =
    \exp\!\left[\frac{f^{2n}}{2(n!)^2}\,\mathbb{E}[X_n^2]\right],
    \label{eq:gaussian_moment}
\end{equation}
since the moment generating function of a zero-mean Gaussian $X$ with variance $\sigma^2$ is $\mathbb{E}[e^{tX}]=e^{t^2\sigma^2/2}$. Substituting Eq.~\eqref{eq:Xn_var_explicit}:
\begin{align}
    \mathbb{E}_{\eta_n}\!\left[e^{f^n X_n/n!}\right]
    &=
    \exp\!\left[\frac{f^{2n}}{2(n!)^2}\cdot\frac{2(n!)^2}{(2n)!}\int K_{2n}\right]
    \nonumber\\
    &=
    \exp\!\left[\frac{f^{2n}}{(2n)!}\int K_{2n}\right]
    \nonumber\\
    &=
    \exp\!\left[\alpha_{2n}\,f^{2n}\right],
    \label{eq:level_n_result}
\end{align}
where we define the integrated cumulant coefficient
\begin{equation}
    \alpha_{2n} := \frac{1}{(2n)!}\int_0^\beta d\tau_1\cdots d\tau_{2n}\;K_{2n}(\tau_1,\ldots,\tau_{2n}).
    \label{eq:alpha2n_def}
\end{equation}
Everything cancels beautifully: the $(n!)^2$ from the HS covariance formula eats the $(n!)^2$ in the denominator from the coupling $f^n/n!$ squared, and what remains is the natural cumulant prefactor $1/(2n)!$. 

\subsection{The theorem}

Combining all levels of the hierarchy, the reduced equilibrium operator becomes

\medskip
\noindent\fbox{\parbox{0.95\columnwidth}{%
\textbf{Commuting Sector Theorem.} \emph{If\/ $[H_S,f]=0$ and all odd cumulants vanish, then}
\begin{equation}
    \bar\rho_S(\beta)
    =
    Z_B\,\exp\!\left(-\beta H_S + \sum_{n=1}^{\infty}\alpha_{2n}\,f^{2n}\right),
    \label{eq:commuting_theorem}
\end{equation}
\emph{with $\alpha_{2n}$ given by Eq.~\eqref{eq:alpha2n_def}. Equivalently, the Hamiltonian of mean force is}
\begin{equation}
    H_{\mathrm{MF}}(\beta) = H_S - \frac{1}{\beta}\sum_{n=1}^{\infty}\alpha_{2n}(\beta)\,f^{2n}.
    \label{eq:HMF_commuting_full}
\end{equation}
}}
\medskip

\noindent Several points are worth emphasising:

\begin{enumerate}
    \item \textbf{Exactness.} This is not a perturbative result. It holds for arbitrary coupling strength, arbitrary temperature, and arbitrary (even non-Gaussian) bath statistics. No truncation of the cumulant series is performed; the theorem holds order by order, and resumming is exact.

    \item \textbf{No PSD requirement.} Nowhere did we assume that any cumulant kernel $K_{2n}$ is positive semi-definite. The HS transformation is valid for indefinite and complex kernels, with convergence guaranteed by contour rotation. In physical terms: baths with sub-Gaussian tails (like bounded spin baths) generically produce $K_4 < 0$, and the construction handles this without difficulty.

    \item \textbf{Contour rotation suffices.} For complex $\alpha_{2n}$ (arising from complex integrated cumulants), the exponential in Eq.~\eqref{eq:commuting_theorem} is well-defined as a matrix function in any eigenbasis of $f$. Hermiticity of $\bar\rho_S$ is ensured by the Hermiticity constraints on the cumulants of a self-adjoint operator $B$.

    \item \textbf{Gaussian limit.} Setting $K_n = 0$ for $n \geq 3$ gives $\alpha_2 = \frac{1}{2}\int\!\!\int K_2(\tau,\tau')\,d\tau\,d\tau' = \frac{1}{2}C(\beta)$, reproducing $-\lambda f^2$ with $\lambda = C(\beta)/(2\beta)$---the reorganisation energy result of the companion paper.
\end{enumerate}

The theorem tells us that the entire bath influence, in the commuting sector, enters as a renormalised polynomial in $f$. The coefficients $\alpha_{2n}$ are nothing but the integrated cumulants of the bath coupling operator, weighted by the natural combinatorial factor $1/(2n)!$. Each cumulant order contributes one new power of $f^2$ to the Hamiltonian of mean force.

\section{\label{sec:numerics}Numerical Validation (Commuting Sector)}

We now validate the nested Hubbard-Stratonovich construction against exact diagonalization (ED) benchmarks. To demonstrate the flexibility of the framework, we consider three distinct commuting models: a multi-qubit system with sector-dependent interactions, a Gaussian qutrit clock model, and a qutrit coupled to a non-Gaussian finite transverse-spin bath. In all cases, we compare the analytic predictions (derived in Sec.~\ref{sec:commuting}) and stochastic sampling results against the exact reduced density matrix.

For all benchmarks, the reference state $\rho_S^{\mathrm{ED}}(\beta)$ is computed by exact diagonalization of the total system-bath Hamiltonian:
\begin{equation}
\rho_S^{\mathrm{ED}}(\beta)=\frac{\Tr_B\,e^{-\beta H_{\mathrm{tot}}}}{\Tr\,e^{-\beta H_{\mathrm{tot}}}}.
\label{eq:v2_rho_ed}
\end{equation}
From this, we define the exact mean-force operator $H_{\mathrm{MF}}^{\mathrm{ED}}=-\beta^{-1}\log \rho_S^{\mathrm{ED}}+\mathrm{const}$. By construction, rebuilding $\rho$ from $H_{\mathrm{MF}}^{\mathrm{ED}}$ is exact up to numerical precision, as quantified by the trace distance \(D_{\mathrm{HMF}}=\tfrac12\|\rho_S^{\mathrm{ED}}-\mathcal N[e^{-\beta H_{\mathrm{MF}}^{\mathrm{ED}}}]\|_1\approx 0\). This diagnostic is monitored throughout our benchmarks to ensure that any observed errors are due to the approximations within the nested HS framework, rather than inconsistencies in the reference state.

\subsection{Multi-qubit magnetisation sectors}

Our first test considers an $N$-qubit system where the interaction is determined by the total magnetisation \(S_z=\sum_{i=1}^N \sigma_i^z\). The environmental influence is chosen to induce a sector distribution \(p(m)\propto g(m)\,\exp\!\left(k_2 m^2 + k_4 m^4\right)\), where $g(m)$ is the binomial degeneracy of the magnetisation sector $m\in\{-N,-N+2,\dots,N\}$. By treating $k_2$ and $k_4$ as effective cumulants, we can subject the nested HS framework to a controlled stress test where the exact result is known analytically.

We examine two distinct regimes. In the positive semi-definite (PSD) branch ($k_4 > 0$), the quartic influence kernel is stable and standard real-valued auxiliary fields suffice. In the non-PSD branch ($k_4 < 0$), the kernel is inverted, requiring the quartic auxiliary field to effectively integrate along the imaginary axis (contour rotation) to ensure convergence. We compare exact analytic sector probabilities against HS Monte Carlo in both branches. Diagnostics include trace distance, max absolute error, and imaginary leakage \(\sum_m |\Im w_m| \big/ \sum_m |\Re w_m|\).

Figure~\ref{fig:nhs_synth_nonpsd_v2} compares the exact sector probabilities with those obtained from the nested HS Monte Carlo sampling. Panels (A) and (B) demonstrate that the stochastic sampling faithfully reproduces the analytic distribution in both the PSD and non-PSD regimes. The convergence of the trace distance with sample count $M$, shown in Panel (C), confirms that the contour rotation method for negative quartic cumulants yields stable, unbiased estimates with standard Monte Carlo scaling. This reinstates the multi-qubit magnetisation stress test while validating contour-rotated HS sampling before the finite-bath qutrit benchmarks.

\begin{figure*}[t]
    \centering
    \IfFileExists{../../simulations/results_v2/figures/nhs_synth_nonpsd_v2.pdf}{
    \includegraphics[width=\textwidth]{../../simulations/results_v2/figures/nhs_synth_nonpsd_v2.pdf}
    }{
    \fbox{\parbox{0.95\textwidth}{\centering
    Placeholder: run \texttt{plot\_nested\_hs\_suite\_v2.py} to generate\\
    \texttt{simulations/results\_v2/figures/nhs\_synth\_nonpsd\_v2.pdf}.}}
    }
    \caption{\label{fig:nhs_synth_nonpsd_v2}
    \textbf{Multi-qubit magnetisation sectors.}
    Commuting $S_z$-sector benchmark with quartic cumulants.
    (A) PSD regime ($k_4>0$): analytic vs Monte Carlo sector probabilities.
    (B) Non-PSD regime ($k_4<0$): contour-rotated Monte Carlo vs analytic.
    (C) Trace-distance convergence with sample count $M$.
    }
\end{figure*}

\subsection{Qutrit clock phase model}

To rigorously test the handling of complex-valued cumulants, we employ a qutrit clock model where the quartic cumulant carries a complex phase, \(\kappa_4=r\,e^{i\phi}\), while the quadratic cumulant satisfies \(\kappa_2=\kappa_4^\ast\). This setup breaks time-reversal symmetry in the influence functional and requires the auxiliary fields to sample complex-valued configurations. We sample populations versus $\phi/\pi$ for several correlation lengths $\tau_c$.

For each $(\phi,\tau_c)$, we compare analytic phase-dependent populations to discretised functional HS Monte Carlo. To reduce Monte Carlo jaggedness, each $(\phi,\tau_c)$ point is estimated from replica-averaged sampling with a fixed total sample budget. This benchmark isolates phase handling and complex auxiliary-field sampling quality.

Figure~\ref{fig:nhs_phase_clock_v2} presents the population dynamics as a function of the phase $\phi$. Panel (A) shows smooth analytic phase curves with Monte Carlo overlays (representative $\tau_c$) for direct visual matching. Panel (B) quantifies the error, plotting the trace distance error versus phase for each $\tau_c$ with replica uncertainty bands, confirming stable phase tracking.

\begin{figure*}[t]
    \centering
    \IfFileExists{../../simulations/results_v2/figures/nhs_cng_phase_clock_v2.pdf}{
    \includegraphics[width=\textwidth]{../../simulations/results_v2/figures/nhs_cng_phase_clock_v2.pdf}
    }{
    \fbox{\parbox{0.95\textwidth}{\centering
    Placeholder: run \texttt{plot\_nested\_hs\_suite\_v2.py} to generate\\
    \texttt{simulations/results\_v2/figures/nhs\_cng\_phase\_clock\_v2.pdf}.}}
    }
    \caption{\label{fig:nhs_phase_clock_v2}
    \textbf{Qutrit clock phase model.}
    (A) Phase-resolved populations: smooth analytic curves with Monte Carlo overlays.
    (B) Trace-distance error $D(\mathbf p^{\mathrm{MC}},\mathbf p^{\mathrm{analytic}})$ vs phase, shown for each correlation length $\tau_c$ with replica uncertainty bands.
    }
\end{figure*}

\subsection{Gaussian finite-bath benchmark}

As a baseline for the finite-bath simulations, we consider a qutrit system with Hamiltonian \(H_S=\mathrm{diag}(0,0.9,1.8)\) coupled via \(f=\mathrm{diag}(0,1,2)\) to a discretised Ohmic bosonic bath. The bath Hamiltonian is \(H_B=\sum_k\omega_k(a_k^\dagger a_k+\tfrac12)\), and the interaction is \(H_I=f\otimes\sum_k c_k x_k\), with \(x_k=(a_k+a_k^\dagger)/\sqrt{2\omega_k}\). In this Gaussian limit, the exact HMF solution involves a simple reorganisation energy shift \(-\lambda_{\mathrm{disc}}f^2\), where \(\lambda_{\mathrm{disc}}=\sum_k c_k^2/(2\omega_k^2)\).

At each $(\beta,g)$, we compare three quantities: the exact result from finite-bath ED, the commuting analytic prediction \(\rho_S^{\mathrm{an}}\propto e^{-\beta(H_S-\lambda_{\mathrm{disc}}f^2)}\), and scalar and path HS estimators. We also report $D_{\mathrm{HMF}}$ from exact HMF reconstruction. Primary observables are $p_2=\langle 2|\rho_S|2\rangle$ and trace distances \(D(\rho_a,\rho_b)=\tfrac12\|\rho_a-\rho_b\|_1\). We additionally monitor \(\lambda_{\mathrm{est}}= \frac{E_2-E_0+\beta^{-1}\ln(p_2/p_0)}{f_2^2-f_0^2}\) for $\beta$-invariance and $g^2$ scaling.

Figure~\ref{fig:nhs_cg_v2} summarizes the results. Panel (A) confirms that all three methods agree on the equilibrium populations. Panel (B) analyzes the convergence of the discrete-bath diagonalization to the continuum analytic result as the number of bath modes increases, and shows Monte Carlo scaling. The exact-HMF reconstruction diagnostic is numerically at machine precision across the grid. The HS estimator residuals are shown to be consistent with the expected statistical noise, validating the Gaussian baseline of our code before introducing non-Gaussianity.

\begin{figure*}[t]
    \centering
    \IfFileExists{../../simulations/results_v2/figures/nhs_cg_1_v2.pdf}{
    \includegraphics[width=\textwidth]{../../simulations/results_v2/figures/nhs_cg_1_v2.pdf}
    }{
    \fbox{\parbox{0.95\textwidth}{\centering
    Placeholder: run \texttt{plot\_nested\_hs\_suite\_v2.py} to generate\\
    \texttt{simulations/results\_v2/figures/nhs\_cg\_1\_v2.pdf}.}}
    }
    \caption{\label{fig:nhs_cg_v2}
    \textbf{Gaussian qutrit benchmark.}
    (A) Coupled-level population $p_2$ vs inverse temperature: ED, analytic prediction, and HS estimators.
    (B) ED-to-analytic cutoff convergence with HS residual baselines and Monte Carlo scaling inset.
    }
\end{figure*}

\subsection{Non-Gaussian transverse-spin bath}

Finally, we address the core challenge: a non-Gaussian environment. We couple the qutrit system to a finite bath of $N_B$ spins via a transverse interaction \(H_I=g\,f\otimes B\), where \(B=\sum_{j=1}^{N_B} c_j \sigma_j^z\), and the bath Hamiltonian is \(H_B=\sum_{j=1}^{N_B}\omega_j\sigma_j^x\). This model generates non-trivial odd and even cumulants at all orders.

For each $(\beta,g)$, we compute $\rho_S^{\mathrm{ED}}$ from exact diagonalisation and extract cumulant coefficients $\alpha_2,\alpha_4,\alpha_6$ from derivatives of $\log Z_B(\theta)$. We then compare the nested HS truncation hierarchy, specifically $K_2$ (Gaussian), $K_2{+}K_4$, and stability-gated $K_2{+}K_4{+}K_6$ truncations. The sixth-order term is included only when the derivative-stability diagnostic satisfies \(\epsilon_6^{\mathrm{stab}}\le 0.35\); otherwise $\alpha_6$ is clipped to zero. Primary residuals are \(D_{(2)}, D_{(2+4)}, D_{(2+4+6)}\). Renormalisation indicators \(\chi_4=\alpha_4/\alpha_2^2\) and \(\chi_6=\alpha_6/|\alpha_2|^3\) are also monitored, along with exact-HMF reconstruction via $D_{\mathrm{HMF}}$.

Figure~\ref{fig:nhs_cng_v2} compares the exact diagonalization results with the predictions of the truncation ladder. Panel (A) gives ED-vs-truncation population overlays at representative $(\beta,g)$ points, clearly showing that while the Gaussian ($K_2$) approximation captures the qualitative trend, it deviates significantly at strong coupling. The inclusion of the fourth-order cumulant ($K_2+K_4$) substantially reduces this error. Panel (B) shows truncation error versus coupling for low/mid/high $\beta$ slices, comparing $K_2$ and $K_2{+}K_4$. Panel (C) provides a comprehensive validity map in the $(\beta, g)$ plane, plotting the log-ratio of the errors $\log_{10}(D_{K2}/D_{K2+K4})$. Positive values (red/yellow regions) indicate that the quartic correction improves the result, which holds true for the vast majority of the parameter space. Panel (D) analyzes the scaling of the extracted cumulant coefficients $\alpha_n$. By plotting rescaled cumulants $\alpha_2/g^2$ and $\alpha_4/g^4$ with coupling-collapse bands, we clarify that the quartic coefficient is not negligible once coupling scaling is factored out, confirming that statistical non-Gaussianity persists even at weak coupling, distinguishing it from perturbative higher-order effects. The exact-HMF reconstruction remains numerically exact; residuals in (A-D) are truncation error, not an ED/HMF inconsistency.

\begin{figure*}[t]
    \centering
    \IfFileExists{../../simulations/results_v2/figures/nhs_cng_1_v2.pdf}{
    \includegraphics[width=\textwidth]{../../simulations/results_v2/figures/nhs_cng_1_v2.pdf}
    }{
    \fbox{\parbox{0.95\textwidth}{\centering
    Placeholder: run \texttt{plot\_nested\_hs\_suite\_v2.py} to generate\\
    \texttt{simulations/results\_v2/figures/nhs\_cng\_1\_v2.pdf}.}}
    }
    \caption{\label{fig:nhs_cng_v2}
    \textbf{Non-Gaussian finite-spin bath.}
    (A) Qutrit population overlays at representative $(\beta,g)$ for ED and truncation ladder.
    (B) Trace-distance vs coupling for low/mid/high $\beta$ slices, comparing $K_2$ and $K_2{+}K_4$.
    (C) Validity map $\log_{10}(D_{K2}/D_{K2+K4})$ in the $(\beta,g)$ plane; positive values indicate quartic improvement.
    (D) Rescaled cumulants $\alpha_2/g^2$ and $\alpha_4/g^4$ confirm consistent non-Gaussian scaling.
    }
\end{figure*}

This progression of benchmarks---from analytic stress tests to finite-bath spin models---establishes the nested Hubbard-Stratonovich hierarchy as a robust, convergent, and numerically stable method for constructing Mean Force Hamiltonians in non-Gaussian environments.

Non-commuting validations are intentionally deferred to the \textit{what\_rules\_equilibrium} paper; this manuscript version is strictly commuting-sector.

\input{sections_v3/05_consequences_v3.tex}
\section{Demonstration: Cumulant Attachment in N-level Systems}
\label{sec:algebraic_demo}

The algebraic closure principle derived in Sec.~\ref{sec:consequences} makes a strong prediction: the ability of a non-Gaussian bath to deform the system Hamiltonian is strictly limited by the algebra of the coupling operator $f$. In this section, we test this prediction directly by comparing two $N$-level clock models coupled to the same non-Gaussian bath.

Consider an $N$-level system with the clock coupling operator $f = \mathrm{diag}(1, \omega, \dots, \omega^{N-1})$, where $\omega = e^{2\pi i/N}$. The system Hamiltonian is $H_S = J(f + f^\dagger)$. The spin bath is identical to the one in Sec.~\ref{sec:numerics}, characterized by significant non-Gaussian spectral content ($\alpha_4 \neq 0$).

\subsection{Automatic Triviality ($N=4$)}
To isolate the algebraic filtering effect from dynamical complexity, we consider a system Hamiltonian $H_S$ that commutes with the coupling operator $f$, i.e., $[H_S, f] = 0$. This corresponds to a pure dephasing model where the bath modulates the energy levels without inducing transitions.
For $N=4$, the eigenvalues of $f$ are fourth roots of unity, so $f^4 = I$.
Since the cumulant expansion in the commuting limit effectively adds terms proportional to $\alpha_n f^n$ to the Hamiltonian, the fourth-order term $\alpha_4 f^4$ becomes a scalar $\alpha_4 I$. This shifts all energy levels uniformly, leaving the population statistics $p_k \propto \langle k | e^{-\beta H_{eff}} | k \rangle$ invariant.
Consequently, the $K_2+K_4$ approximation is \textit{exactly} identical to the Gaussian $K_2$ result.

Figure~\ref{fig:algebraic_closure}(A) shows the populations $P_0$ and $P_1$ extended to $g=1.2$. The Gaussian approximation ($K_2$, dashed blue) deviates significantly from the exact result (solid black/gray).
Crucially, the $K_2+K_4$ prediction (dotted red) lies \textbf{exactly} on top of the Gaussian curve for both levels, confirming the algebraic invisibility of the quartic cumulant. The error is corrected by the sixth-order cumulant ($K_2+K_4+K_6$, dot-dashed green) for moderate couplings ($g \lesssim 0.8$), but eventually deviates as the perturbative series approaches its radius of convergence.

\subsection{Cumulant Attachment ($N=5$)}
For $N=5$, $f^4$ is not proportional to $I$ or $f^2$. The fourth-order term $\alpha_4 f^4$ modifies the energy landscape non-uniformly.
Figure~\ref{fig:algebraic_closure}(B) shows that the $K_2+K_4$ approximation (dotted red) immediately separates from the Gaussian baseline (dashed blue) for all levels and corrects the error, moving closer to the exact result before the series expansion limits are reached.

This comparison demonstrates "algebraic closure": when the system algebra is closed at a lower order ($f^4=I$), high-order non-Gaussian information is structurally filtered out. Convergence requires that the cumulants attach to linearly independent operator slots in the effective Hamiltonian.

\begin{figure}[t]
    \centering
    \IfFileExists{../../simulations/results_v2/figures/nhs_algebraic_closure.pdf}{
    \includegraphics[width=\columnwidth]{../../simulations/results_v2/figures/nhs_algebraic_closure.pdf}
    }{
    \includegraphics[width=\columnwidth]{example-image-a}
    }
    \caption{\label{fig:algebraic_closure}
    \textbf{Algebraic Filtering ($[H_S, f]=0$).}
    Population of first two energy levels ($P_0$ dark, $P_1$ light) vs coupling $g$.
    (A) $\mathbf{N=4}$: Since $f^4 = I$, the quartic correction is a scalar shift and is invisible ($K2+K4$ overlaps $K2$ for both levels). The error is corrected by $K_6$ up to the convergence radius ($g \approx 0.8$).
    (B) $\mathbf{N=5}$: $f^4$ is non-trivial. $K_4$ explicitly corrects the Gaussian error (Red curve separates from Blue).    }
\end{figure}

\input{sections_v3/06_discussion_v3}

\vspace{1em}
\appendix
\appendix

\section{\label{app:simulations}Simulation Implementation Details}

This appendix describes the discretised functional sampling protocol used in Sec.~\ref{sec:numerics}. All source code is available in the project repository.

\subsection{Discretisation and kernel construction}

Imaginary time $[0,\beta]$ is discretised into $N$ points $\tau_i = (i + \frac{1}{2})\Delta\tau$ with $\Delta\tau = \beta/N$. The second cumulant is represented by an $N\times N$ exponential kernel:
\begin{equation}
    K_{2,ij}^{\text{base}} = \exp\!\left(-\frac{|\tau_i - \tau_j|}{\tau_c}\right),
    \label{eq:K2_base}
\end{equation}
with correlation length $\tau_c$. This kernel is strictly positive definite for all $\tau_c > 0$; its Cholesky factor $L$ ($K_2^{\text{base}} = LL^T$) is computed with a jitter fallback for numerical stability.

The kernel amplitude is calibrated to match the target integrated second cumulant $\kappa_2$. The scalar contraction $S = \Delta\tau^2\,\mathbf{1}^T K_2^{\text{base}}\,\mathbf{1}$ determines the required complex amplitude: $A = 2\kappa_2/S$. We define the complex square root
\begin{equation}
    \sqrt[\text{c}]{z} = \sqrt{|z|}\,e^{i\arg(z)/2},
    \label{eq:complex_sqrt}
\end{equation}
which handles contour rotation for non-PSD or complex-valued covariances.

\subsection{Monte Carlo field generation}

At each Monte Carlo sample, two independent fields are drawn:

\textbf{Gaussian field (second cumulant).} Draw $\mathbf{z}\sim\mathcal{N}(0,I_N)$ and form $\boldsymbol{\xi} = \sqrt[\text{c}]{A}\,L\,\mathbf{z}$. The scalar contraction is $X = \Delta\tau\sum_i\xi_i$, which is a zero-mean complex-valued scalar with covariance $\mathbb{E}[X^2] = 2\kappa_2$.

\textbf{Bilocal field (fourth cumulant).} The fourth cumulant contributes through a rank-1 bilocal field whose scalar contraction is $Y$. Draw $z\sim\mathcal{N}(0,1)$ and set $Y = \sqrt[\text{c}]{\text{Var}(Y)}\,z\,N^2\Delta\tau^2$, with $\text{Var}(Y) = 8\kappa_4/(N^4\Delta\tau^4)$. For $\kappa_4 < 0$ (non-PSD), $\sqrt[\text{c}]{\text{Var}(Y)}$ is purely imaginary, rotating the field contour automatically.

\subsection{Weight estimation}

For the qutrit model, the unnormalised weight for eigenvalue $\lambda_k$ given sample $(X, Y)$ is
\begin{equation}
    w_k = \exp\!\Big[-\beta J(\lambda_k + \lambda_k^*) + \lambda_k X + \tfrac{1}{2}\lambda_k^2 Y\Big].
    \label{eq:weight_qutrit}
\end{equation}
The populations are estimated as $p_k = \re\bar{w}_k / \sum_j \re\bar{w}_j$, where $\bar{w}_k$ is the sample mean of $w_k$ over $M$ Monte Carlo draws.

For the multi-qubit model, the analogous weight in sector $m$ is $w_m = g(m)\exp(mX + m^2Y/2)$.

\subsection{Common random numbers and vectorisation}

To isolate the effect of parameter variation from Monte Carlo noise, all phase-sweep and correlation-length comparisons use common random numbers (CRN): the same draws $\mathbf{z}$ and $z$ are shared across all parameter values in a given figure panel. This dramatically reduces the variance of \emph{differences} between Monte Carlo estimates at different parameters, enabling clean collapse diagnostics.

The implementation is vectorised: all $M$ samples of $X$ and $Y$ are computed as vector operations, and the weight matrix $W_{mk}$ is formed via an outer product $W = \exp(X\otimes\lambda + \frac{1}{2}Y\otimes\lambda^2)$, avoiding per-sample loops.

\subsection{Error metrics}

\textbf{Population error:} $\varepsilon_{\max} = \max_k |p_k^{\text{MC}} - p_k^{\text{exact}}|$.

\textbf{Trace distance:} $D(p,q) = \frac{1}{2}\sum_k |p_k - q_k|$.

\textbf{Imaginary leakage:} $\ell = \max_k |\im\bar{w}_k|/|\re\bar{w}_k|$, which measures the magnitude of imaginary residuals. In exact arithmetic, Hermiticity of $\bar\rho_S$ guarantees $\ell = 0$ after normalisation; nonzero $\ell$ reflects finite-sample fluctuations and is confirmed to decrease as $1/\sqrt{M}$.


\bibliography{references_new}

\end{document}
