\documentclass[aps,pra,twocolSumn,superscriptaddress,showpacs]{revtex4-2}

\usepackage{graphicx}
\usepackage{amsmath}
\usepackage{amssymb}
\usepackage{hyperref}
\usepackage{mathtools}
\usepackage{bbm}

\newcommand{\Tr}{\mathrm{Tr}}
\newcommand{\re}{\mathrm{Re}\,}
\newcommand{\im}{\mathrm{Im}\,}
\setcounter{secnumdepth}{3}

\begin{document}

\title{Nested Hubbard--Stratonovich Representations\\for Non-Gaussian Influence Functionals}
\author{Gerard McCaul}
% \affiliation{Affiliation}
\date{\today}

\begin{abstract}
The reduced equilibrium state of a quantum system coupled to a general environment is determined by the imaginary-time influence functional, whose cumulant structure encodes the full bath statistics. For Gaussian baths, a single Hubbard--Stratonovich transformation decouples the bilocal influence kernel. We show that the complete cumulant hierarchy---including all even-order connected correlators---admits an exact, constructive representation as a nested sequence of Gaussian auxiliary fields. At each order $2n$, the $n$-local composite of system coupling operators is paired with a Gaussian field whose covariance is fixed by the symmetry-projected cumulant kernel. The construction requires neither positive semi-definiteness nor boundedness of the cumulant kernels; contour rotation of the auxiliary fields suffices. In the commuting sector $[H_S,f]=0$, the entire nested hierarchy re-averages analytically, yielding an exact closed-form Hamiltonian of mean force as a polynomial in the coupling operator: $H_{\mathrm{MF}} = H_S - \sum_{n\geq 1}\alpha_{2n}\,f^{2n}$, with coefficients determined by integrated cumulants. We validate the construction numerically via discretised functional sampling for a qutrit clock model and a multi-qubit magnetisation model, demonstrating agreement with exact analytics across PSD and non-PSD parameter regimes.
\end{abstract}

\maketitle

\section{\label{sec:intro}Introduction}

The Hamiltonian of mean force (HMF) is the operator whose Gibbs form reproduces the reduced equilibrium state of a quantum system coupled to an environment~\cite{campisiFluctuationTheoremArbitrary2009,talknerColloquiumStatisticalMechanics2020,trushechkinOpenQuantumSystem2022,seifertFirstSecondLaw2016}. For a total Hamiltonian $H = H_S + H_B + f\otimes B$, the unnormalised reduced equilibrium operator is $\bar\rho_S(\beta) = \Tr_B\,e^{-\beta H}$, and the HMF is defined by $e^{-\beta H_{\mathrm{MF}}(\beta)} = \bar\rho_S(\beta)/Z_B(\beta)$. At weak coupling, $H_{\mathrm{MF}}$ reduces to the bare system Hamiltonian; at finite coupling it acquires temperature-dependent operator content that is not captured by a simple renormalisation~\cite{hanggiFiniteQuantumDissipation2008,ingoldSpecificHeatAnomalies2009,espositoNatureHeatStrongly2015,rivasStrongCouplingThermodynamics2020}.

The influence-functional formalism provides a natural route to the HMF~\cite{feynmanTheoryGeneralQuantum1963a,caldeiraQuantumTunnellingDissipative1983a,grabertQuantumBrownianMotion1988}. For Gaussian baths with linear coupling, the bath degrees of freedom can be integrated out exactly, producing a bilocal imaginary-time kernel whose Hubbard--Stratonovich (HS) decoupling~\cite{hubbardCalculationPartitionFunctions1959a,stratonovich1957QDistro} represents the reduced equilibrium state as a Gaussian average over quenched propagators~\cite{mccaulPartitionfreeApproachOpen2017c,mccaulDrivingSpinbosonModels2018a,stockburgerExactNumberRepresentation2002}. In the commuting sector $[H_S,f]=0$, the time-ordering obstruction vanishes and the Gaussian average can be performed in closed form, yielding an exact HMF~\cite{campisiTalknerHanggi2009Solvable}. This programme was carried out explicitly in Ref.~\cite{mccaulPartitionfreeApproachOpen2017c} for harmonic baths, where the resulting HMF contains a single bath-induced correction $-\lambda f^2$ with $\lambda$ the reorganisation energy.

However, many physically important environments are not Gaussian. Spin baths~\cite{ProkofevStamp2000}, collections of two-level fluctuators, and any bounded bath generate non-Gaussian statistics in the collective coupling variable~\cite{HsiehCao2018I,HsiehCao2018II}. When the individual couplings remain $\mathcal{O}(1)$ as the bath grows, the central limit theorem does not apply and connected correlators of order $n\geq 3$ contribute at leading order. The influence functional then contains higher-order multilocal terms beyond the Gaussian kernel, and the single HS transformation is insufficient.

Several approaches address non-Gaussian effects. Cluster and cumulant expansions have been applied to spin-bath dynamics~\cite{SuarezSilbey1991,Makri1999,YingEtAl2024}, and the polaron and reaction-coordinate frameworks absorb part of the interaction into a renormalised system~\cite{leggettDynamicsDissipativeTwostate1987,HsiehCao2018II}. Functional-integral treatments of non-Gaussian noise have appeared in real-time settings~\cite{FunoIshizaki2024,HalataeiEtAl2025}, and recent work has examined the structure of the HMF beyond harmonic baths~\cite{burkeStructureHamiltonianMean2024,duGeneralizedHamiltonianMeanforce2025a}. What is missing is a constructive, exact representation of the full non-Gaussian influence functional that reduces to a tractable operator expression for the HMF.

In this paper we provide such a construction. We show that the complete imaginary-time cumulant expansion of the influence functional---including all even-order connected correlators---admits an exact representation as a nested hierarchy of Gaussian auxiliary fields. At order $2n$, the $n$-local composite of system coupling operators is paired with a Gaussian field whose covariance is determined by the symmetry-projected cumulant kernel $K_{2n}^{(n|n)}$. The construction is not perturbative: it is an exact identity at each order, valid for kernels that are indefinite, non-positive-semi-definite (non-PSD), or complex. Contour rotation of the auxiliary fields mode-by-mode suffices to define the Gaussian integrals.

In the commuting sector, the entire nested hierarchy re-averages analytically. Each auxiliary field produces a scalar Gaussian contraction, and the resulting series exponentiates to give an exact closed-form HMF:
\begin{equation}
    H_{\mathrm{MF}}(\beta) = H_S - \sum_{n\geq 1} \alpha_{2n}(\beta)\,f^{2n},
    \label{eq:HMF_preview}
\end{equation}
with $\alpha_{2n} = \frac{1}{(2n)!}\int_0^\beta d\tau_1\cdots d\tau_{2n}\,K_{2n}(\tau_1,\ldots,\tau_{2n})$. This is the main result of the paper.

The paper is structured as follows. Section~\ref{sec:cumulants} defines the reduced equilibrium operator and derives the ordered cumulant expansion of the influence functional. Section~\ref{sec:nested_hs} constructs the nested HS representation for arbitrary even-order cumulants and states the general theorem. Section~\ref{sec:commuting} proves the commuting-sector theorem. Section~\ref{sec:numerics} validates the construction via discretised functional sampling for a qutrit clock model and a multi-qubit magnetisation model. Section~\ref{sec:consequences} analyses the consequences of the polynomial HMF structure for qubits, qutrits, and integrable spin chains.

\input{sections/01_setup_cumulants}
\input{sections/02_nested_hs}
\input{sections/03_commuting_sector}
\section{\label{sec:numerics}Numerical Validation}

We validate the nested HS construction and the commuting-sector theorem with two numerical demonstrations. Both operate entirely in the commuting sector $[H_S,f]=0$ and compare discretised functional HS sampling against exact analytic predictions. The first uses a qutrit clock model to stress-test the non-PSD regime; the second uses a multi-qubit magnetisation model to demonstrate convergence in a many-body setting. Full implementation details are in Appendix~\ref{app:simulations}.

\subsection{Qutrit clock model}

The system is a qutrit with clock coupling $f = Z_3 = \mathrm{diag}(1,\omega,\omega^2)$, $\omega = e^{2\pi i/3}$, and bare Hamiltonian $H_S = J(f + f^\dagger)$. This is a natural test bed because the clock algebra closes at order three ($f^3 = I$), so the Gaussian and quartic cumulants contribute to distinct operator channels (see Sec.~\ref{sec:consequences}).

We parametrise the bath by complex integrated cumulants $\kappa_4 = r\,e^{i\phi}$ and $\kappa_2 = \kappa_4^*$ (ensuring Hermiticity), and sweep the phase $\phi\in[-\pi,\pi]$ at fixed amplitude $r = 0.45$, $J = 0.6$, $\beta = 1$. For each $\phi$, the analytic prediction follows directly from Eq.~\eqref{eq:commuting_theorem}:
\begin{equation}
    p_k = \frac{w_k}{\sum_j w_j},\quad
    w_k = \exp\!\big[-\beta J(\lambda_k+\lambda_k^*) + \kappa_2\lambda_k^2 + \kappa_4\lambda_k\big],
    \label{eq:qutrit_analytic}
\end{equation}
where $\lambda_k = \omega^k$ ($k=0,1,2$) are the eigenvalues of $f$.

\begin{figure}[t]
    \centering
    \includegraphics[width=\columnwidth]{../../simulations/results/figures/fig1_populations_collapse.pdf}
    \caption{\label{fig:qutrit_validation}Populations versus cumulant phase for the qutrit clock model. Solid lines: analytic prediction from Eq.~\eqref{eq:qutrit_analytic}. Markers: Monte Carlo estimates from full discretised functional HS sampling at three different correlation lengths $\tau_c$. Agreement is quantitative across the entire phase range, confirming correlation-length independence: only the integrated cumulants matter.}
\end{figure}

The Monte Carlo estimates are obtained by discretised functional sampling: an $N$-point imaginary-time grid carries a Gaussian field $\xi$ with an exponential kernel $K_2^{\text{base}}(\tau,\tau') = e^{-|\tau-\tau'|/\tau_c}$, calibrated so that the scalar contraction gives $\text{Var}(X) = 2\kappa_2$, plus an independent bilocal scalar $s$ calibrated for the quartic contraction $\text{Var}(Y) = 8\kappa_4$. For non-PSD parameters (e.g.\ near $\phi = 0.9\pi$), the complex variances are handled by drawing real Gaussian deviates and multiplying by $\sqrt{|\text{Var}|}\,e^{i\arg(\text{Var})/2}$---contour rotation in action.

Figure~\ref{fig:qutrit_validation} shows the key result: Monte Carlo populations agree quantitatively with the analytic curve for all $\phi$, with three different $\tau_c$ values collapsing onto the same universal curve. This demonstrates that the detailed time structure of the kernel is irrelevant---only the integrated cumulants enter the commuting-sector HMF.

\begin{figure}[t]
    \centering
    \includegraphics[width=\columnwidth]{../../simulations/results/figures/fig2_crn_replicas.pdf}
    \caption{\label{fig:qutrit_diagnostics}Left: CRN error curves for three correlation lengths (common random numbers). Right: replica error statistics (12 independent replicas, $M = 50{,}000$ samples each), showing mean $\pm 1\sigma$ bands. The error is uniformly $\mathcal{O}(10^{-3})$ and independent of $\tau_c$.}
\end{figure}

Figure~\ref{fig:qutrit_diagnostics} provides quantitative error diagnostics. With common random numbers (CRN), the maximum population error is uniformly $\lesssim 10^{-3}$ for $M = 120{,}000$ samples. Replica statistics confirm that the error is Gaussian-distributed across independent runs, with no systematic bias.

\begin{figure}[t]
    \centering
    \includegraphics[width=\columnwidth]{../../simulations/results/figures/fig3_M_scaling.pdf}
    \caption{\label{fig:qutrit_scaling}$\sqrt{M}$-rescaled error versus phase. The collapse of all three curves confirms the expected $1/\sqrt{M}$ Monte Carlo convergence rate.}
\end{figure}

Figure~\ref{fig:qutrit_scaling} shows the $\sqrt{M}$ scaling collapse: plotting $\sqrt{M}\times\varepsilon$ for $M\in\{20{,}000, 80{,}000, 320{,}000\}$ gives curves that coincide, confirming the standard $1/\sqrt{M}$ convergence expected for a well-defined Monte Carlo estimator.

\subsection{Multi-qubit magnetisation model}

\begin{figure*}[t]
    \centering
    \includegraphics[width=\textwidth]{../../simulations/results/figures/fig_multiqubit.pdf}
    \caption{\label{fig:multiqubit}Multi-qubit $S_z$ model ($N=4$ qubits). Left: PSD case ($k_4 = 0.003 > 0$). Centre: non-PSD case ($k_4 = -0.003 < 0$, contour-rotated sampling). Right: trace-distance convergence $D \propto 1/\sqrt{M}$ in both regimes. Bar heights compare analytic (shaded) and MC (solid) sector probabilities.}
\end{figure*}

The second test uses $N = 4$ qubits with $H_S = 0$ and coupling $f = S_z = \sum_{i=1}^N Z_i$, so that the system is block-diagonal in magnetisation sectors $m \in \{-4,-2,0,2,4\}$ with degeneracies $g(m) = \binom{4}{(4+m)/2}$. The bath deformation is $F(S_z) = k_2\,S_z^2 + k_4\,S_z^4$. The exact sector probabilities are
\begin{equation}
    p(m) = \frac{g(m)\,e^{k_2 m^2 + k_4 m^4}}{\sum_{m'} g(m')\,e^{k_2 m'^2 + k_4 m'^4}}.
    \label{eq:multiqubit_exact}
\end{equation}

The HS representation gives $e^{k_2 m^2 + k_4 m^4} = \mathbb{E}_{X,Y}[e^{mX + m^2 Y/2}]$ with $X\sim\mathcal{N}(0,2k_2)$ and $Y\sim\mathcal{N}(0,8k_4)$; for $k_4 < 0$, $Y$ is drawn as $Y = i\sqrt{8|k_4|}\,Z$ with $Z\sim\mathcal{N}(0,1)$.

Figure~\ref{fig:multiqubit} confirms quantitative agreement in both the PSD ($k_4 > 0$) and non-PSD ($k_4 < 0$) regimes. Trace distances are $\lesssim 5\times 10^{-3}$ at $M = 100{,}000$, with $1/\sqrt{M}$ convergence confirmed in the right panel.

\section{\label{sec:consequences}Consequences: Algebraic Closure and the Operator Content of the HMF}

The commuting-sector theorem, Eq.~\eqref{eq:HMF_commuting_full}, gives the Hamiltonian of mean force as a power series in the coupling operator:
\begin{equation}
    H_{\mathrm{MF}} = H_S - \frac{1}{\beta}\sum_{n=1}^{\infty}\alpha_{2n}\,f^{2n}.
    \label{eq:HMF_polynomial_recap}
\end{equation}
This is a polynomial in a single operator, determined entirely by the bath cumulants. But the physical content of this polynomial depends crucially on the \emph{algebra} generated by even powers of $f$: how many linearly independent operators appear in the sequence $f^2, f^4, f^6, \ldots\,$? If this algebra closes at finite dimension, then the bath can only renormalise a finite set of operator structures---regardless of how exotic the bath statistics. This is the central structural insight, and it has far-reaching consequences.

\subsection{Algebraic closure and renormalisation}

Define the \emph{even-coupling algebra} as the unital algebra generated by $f^2$:
\begin{equation}
    \mathcal{A}_f = \mathrm{span}\{I,\,f^2,\,f^4,\,f^6,\ldots\}.
    \label{eq:algebra_def}
\end{equation}
By the Cayley--Hamilton theorem, any operator on a $d$-dimensional Hilbert space satisfies a polynomial relation of degree $d$, so the powers $f^2, f^4,\ldots$ eventually become linearly dependent. The algebra $\mathcal{A}_f$ therefore closes at dimension $\dim\mathcal{A}_f \leq d$.

Closure has a sharp physical consequence: \emph{once the algebra closes, the bath cannot generate any new operator structures---it can only renormalise the coefficients of the existing ones}. In the language of the HMF:
\begin{equation}
    H_{\mathrm{MF}} = H_S + \sum_{j=0}^{D-1} c_j(\alpha_{2n},\beta)\,f^{2j},
    \label{eq:HMF_closed_form}
\end{equation}
where $D = \dim\mathcal{A}_f$ and the $c_j$ absorb all contributions from the infinite cumulant series. The environment is completely characterised, from the system's point of view, by $D-1$ real numbers (the coefficients $c_j$ for $j\geq 1$).

This is worth emphasising: a $d$-level system coupled to \emph{any} bath through \emph{any} single coupling operator in the commuting sector has an HMF that is determined by at most $d-1$ parameters. An infinite number of bath cumulants collapse into a finite-dimensional operator deformation.

\subsection{Qubits: automatic triviality}

For a two-level system, any traceless operator $f$ satisfies $f^2 = -(\det f)\,I \propto I$. The algebra closes at $D=1$ (with $f^0 = I$): there are no independent even powers beyond the identity. Consequently, every even power $f^{2n} \propto I$, and the infinite sum in Eq.~\eqref{eq:HMF_polynomial_recap} collapses to a scalar energy shift:
\begin{equation}
    H_{\mathrm{MF}} = H_S + c_0\,I.
    \label{eq:HMF_qubit}
\end{equation}
No matter how exotic the bath statistics, a qubit coupled through a single operator in the commuting sector experiences \emph{no operator deformation at all}---only a global energy shift. The non-Gaussian hierarchy is algebraically invisible, not because the cumulants vanish, but because $f^2$ has already closed the algebra. A qubit coupled to a wildly non-Gaussian spin bath has the same \emph{form} of HMF as one coupled to a harmonic oscillator.

\subsection{Qutrits and higher: non-trivial deformation}

The situation changes qualitatively at $d\geq 3$. For the qutrit clock operator $f = Z_3 = \mathrm{diag}(1,\omega,\omega^2)$ with $\omega = e^{2\pi i/3}$, the minimal polynomial is $f^3 - I = 0$, so $f^2 = f^\dagger$ is independent of $I$ and $f$, but $f^4 = f$ closes the algebra. The HMF becomes
\begin{equation}
    H_{\mathrm{MF}} = H_S + c_0\,I + c_1\,f^2,
    \label{eq:HMF_qutrit}
\end{equation}
where $c_1$ absorbs contributions from the fourth and all higher cumulants. The fourth cumulant generates a genuinely new operator, $f^2 = f^\dagger$, that is not proportional to $f$ or $I$. This is the first case where non-Gaussianity produces qualitatively new physics: the bath \emph{deforms} the effective Hamiltonian, not just its eigenvalues.

More generally, for a $d$-level system with $f$ having $d$ distinct eigenvalues, each cumulant order up to $2\lfloor d/2\rfloor$ contributes an algebraically independent correction before closure forces all higher orders to be linearly dependent.

\subsection{Conditions for a local HMF}

The algebraic structure has deeper implications when we ask: \emph{when can the bath influence be absorbed into a Hamiltonian of mean force at all?} In the general (non-commuting) case, the influence functional produces a nonlocal-in-time self-interaction that cannot be reduced to an operator on the system Hilbert space alone. The HMF exists only when this reduction is possible.

In the commuting sector, the reduction is always possible---the theorem guarantees it. But the result lives in the algebra $\mathcal{A}_f$. If $\mathcal{A}_f$ is a proper subalgebra of the full operator algebra on $\mathcal{H}_S$, then the bath couples to a restricted set of degrees of freedom, and the ``environment'' defined by the HS fields can be thought of as acting on the subalgebra alone.

This perspective connects naturally to the reaction-coordinate formalism~\cite{strathearn2018efficient,nazir2018reaction}, where one absorbs part of the bath into a renormalised system. In our framework, the algebraic closure of $f^{2n}$ determines precisely \emph{which} operators the bath can dress. If the algebra is small (as for qubits), the bath has nowhere to go: it can only shift global scales. If the algebra is large (as for many-body systems with a high-dimensional coupling), new effective interactions emerge at each cumulant order.

The analogy with reaction coordinates is deep: in both cases, part of the environment's influence is absorbed into a redefined system Hamiltonian, with the ``reaction coordinate'' identified by the algebraic structure of the coupling. The difference is that here, the couplings algebra determines the \emph{maximal possible deformation} of the HMF, irrespective of the bath model---it is a property of the system and the coupling alone.

\subsection{Integrable chains: bath as a function of a conserved charge}

The polynomial structure becomes physically transparent---and remarkably powerful---when applied to many-body systems. Consider an integrable spin chain, such as the XXZ model:
\begin{equation}
    H_S = \sum_i \big[J(X_iX_{i+1} + Y_iY_{i+1}) + \Delta\,Z_iZ_{i+1}\big],
    \label{eq:XXZ}
\end{equation}
which conserves total magnetisation $S_z = \sum_i Z_i$. If the bath couples to this conserved quantity, $f = S_z$, then $[H_S, f] = 0$ and the HMF takes the form
\begin{equation}
    H_{\mathrm{MF}} = H_S + F(S_z),\quad F(m) = \sum_{n\geq 1}\alpha_{2n}\,m^{2n}.
    \label{eq:HMF_chain}
\end{equation}

The operator $(S_z)^{2n}$ is a sum of $n$-body diagonal interactions. Expanding:
\begin{align}
    (S_z)^2 &= N\,I + 2\!\sum_{i<j}Z_iZ_j, \label{eq:Sz2}\\[4pt]
    (S_z)^4 &\supset \text{1-, 2-, 3-, and 4-body $Z$ strings}.
    \label{eq:Sz4}
\end{align}
A Gaussian bath ($\alpha_2$ only) generates infinite-range two-body Ising interactions $Z_iZ_j$ across the entire chain. A non-Gaussian bath adds three-body, four-body, and higher diagonal interactions.

But here is the key simplification: because $S_z$ is a conserved charge, the algebra of even polynomials $\mathcal{A}_{S_z}$ acts within each magnetisation sector as a scalar function of $m$. The $2^N$-dimensional operator algebra of the chain reduces to an $N+1$-dimensional function space (one coefficient per sector). No matter how many-body the formal operator expansion looks, the bath influence is captured by a single nonlinear reshaping of the energy landscape:
\begin{equation}
    F(S_z)\big|_m = \sum_{n\geq 1}\alpha_{2n}\,m^{2n}.
    \label{eq:F_sectors}
\end{equation}

This has a satisfying structural consequence: \emph{the bath deformation preserves integrability}. Since $F(S_z)$ commutes with $H_S$, the system $H_S + F(S_z)$ has the same eigenstates as $H_S$---only the energies are shifted, differently in each magnetisation sector. The integrable structure (Bethe ansatz solvability, conservation laws) survives intact. Non-Gaussianity can reshape the spectrum but cannot break integrability.

To summarise: the even-coupling algebra $\mathcal{A}_f$ is the master object. It determines the dimension of the space in which the bath can act, the conditions under which a local HMF exists, and the maximal operator content of the bath-induced deformation. For qubits, the algebra is trivial and the bath is invisible. For many-body systems coupled through a conserved charge, the algebra is large but block-diagonal, and the bath reduces to a nonlinear function of that charge.

\section{\label{sec:discussion}Discussion}

We have shown that the imaginary-time influence functional of a general (non-Gaussian) bath admits an exact, constructive representation as a nested hierarchy of Gaussian auxiliary fields. The construction requires only that the cumulant series exists; it places no conditions on the sign, definiteness, or boundedness of the cumulant kernels. Contour rotation of the auxiliary fields---eigenmode by eigenmode---suffices to define the Gaussian integrals for arbitrary complex or indefinite kernels.

In the commuting sector $[H_S,f]=0$, the entire hierarchy collapses analytically. The Hamiltonian of mean force takes the form of a renormalised polynomial in the coupling operator, with coefficients determined by the integrated cumulants of the bath. This is exact, non-perturbative, and---as the numerical demonstrations confirm---quantitatively reliable even in parameter regimes where the cumulant kernels are strongly non-PSD.

The consequences differ sharply by system dimension. For qubits, the algebraic closure of $f$ renders the non-Gaussian hierarchy invisible: every cumulant contributes to the same two independent operator coefficients. For qutrits and higher-dimensional systems, new operator content enters at each cumulant order, up to the dimension of the coupling algebra. For many-body systems coupled through a conserved charge, the bath generates an arbitrary even function of that charge---reshaping the energy landscape across symmetry sectors without breaking integrability.

Several directions suggest themselves. The nested HS representation extends formally to the non-commuting sector $[H_S,f]\neq 0$, where the auxiliary fields no longer average analytically but can be sampled numerically. This would provide a stochastic formulation of the full non-Gaussian influence functional, complementing existing hierarchical methods~\cite{tanimuraReducedHierarchicalEquations2014}. The construction could also be applied to real-time dynamics, where contour issues become more delicate but the algebraic structure is identical. Finally, the observation that non-Gaussianity preserves integrability when acting through a conserved charge may have implications for quantum thermodynamics in integrable systems---an area where the interplay between conservation laws and bath statistics is not yet well understood.


\input{sections/07_appendix}

\bibliography{../../literature/references_new}

\end{document}
