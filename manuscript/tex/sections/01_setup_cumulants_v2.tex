\section{\label{sec:cumulants}Setup and Cumulant Structure}

We consider
\begin{equation}
H = H_S + H_B + f\otimes B,
\label{eq:v2_h_total}
\end{equation}
with $H_S$ and $f$ acting on the system Hilbert space and $H_B,B$ on the bath Hilbert space. The unnormalised reduced equilibrium operator is
\begin{equation}
\bar\rho_S(\beta)=\Tr_B\,e^{-\beta H},
\qquad
\rho_S(\beta)=\frac{\bar\rho_S(\beta)}{\Tr_S \bar\rho_S(\beta)}.
\label{eq:v2_rho_bar}
\end{equation}
The Hamiltonian of mean force (HMF) is defined by
\begin{equation}
e^{-\beta H_{\mathrm{MF}}}=\bar\rho_S/Z_B,
\qquad
Z_B=\Tr_B\,e^{-\beta H_B}.
\label{eq:v2_hmf_def}
\end{equation}

\subsection{Influence-functional cumulant expansion}

In imaginary time, with
\begin{equation}
B(\tau)=e^{\tau H_B} B e^{-\tau H_B},
\qquad
f(\tau)=e^{\tau H_S} f e^{-\tau H_S},
\end{equation}
we write
\begin{equation}
\bar\rho_S
=
e^{-\beta H_S}\,Z_B
\left\langle
\mathcal T_\tau
\exp\!\left[
\int_0^\beta d\tau\,B(\tau)f(\tau)
\right]
\right\rangle_B.
\label{eq:v2_if_basic}
\end{equation}
The logarithm of the bath average has exact connected-cumulant expansion
\begin{equation}
\log\Big\langle \mathcal T_\tau e^{\int Bf}\Big\rangle_B
=
\sum_{n=1}^{\infty}
\frac{1}{n!}
\int_0^\beta d\tau_1\cdots d\tau_n\,
K_n(\tau_1,\dots,\tau_n)
\prod_{j=1}^{n} f(\tau_j).
\label{eq:v2_cumulant_series}
\end{equation}
Here $K_n$ are bath connected correlators. For baths with vanishing odd moments, only even orders contribute.

\subsection{Source-deformed partition function and integrated cumulants}

For commuting-sector numerics we characterise cumulants through
\begin{equation}
Z_B(\theta)=\Tr_B\,e^{-\beta(H_B+\theta B)},
\qquad
F_B(\theta)=-\beta^{-1}\log Z_B(\theta).
\label{eq:v2_source_partition}
\end{equation}
Time-integrated even cumulants are generated by derivatives at $\theta=0$:
\begin{equation}
\alpha_{2n}(\beta)
=
\frac{1}{\beta(2n)!}
\left.\frac{d^{2n}}{d\theta^{2n}}\log Z_B(\theta)\right|_{\theta=0}.
\label{eq:v2_alpha_def}
\end{equation}
Equation~\eqref{eq:v2_alpha_def} is the coefficient set used directly in the $K_2$, $K_2{+}K_4$, and $K_2{+}K_4{+}K_6$ truncation ladders.
