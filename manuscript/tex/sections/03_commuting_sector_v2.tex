\section{\label{sec:commuting}Commuting Sector Theorem}

Assume
\begin{equation}
[H_S,f]=0.
\label{eq:v2_commuting_assumption}
\end{equation}
Then $f(\tau)=f$ in imaginary time, time ordering is algebraically trivial, and every composite field reduces to a power of $f$.

\subsection{Exact commuting-sector HMF}

With Eq.~\eqref{eq:v2_alpha_def}, the reduced operator becomes
\begin{equation}
\bar\rho_S(\beta)
=
Z_B\,
\exp\!\left[
-\beta H_S
\;+\;
\beta\sum_{n\ge 1}\alpha_{2n} f^{2n}
\right].
\label{eq:v2_commuting_rhobar}
\end{equation}
Hence
\begin{equation}
H_{\mathrm{MF}}(\beta)
=
H_S-\sum_{n\ge 1}\alpha_{2n}(\beta)f^{2n}+\text{const}.
\label{eq:v2_commuting_hmf}
\end{equation}

\subsection{Physical interpretation}

The commuting hierarchy has a clean interpretation:
\begin{enumerate}
\item $\alpha_2$ is the Gaussian reorganisation shift.
\item $\alpha_4,\alpha_6,\dots$ are non-Gaussian renormalisation corrections.
\item Finite spin baths are not ``purely quartic'': higher even cumulants are generically nonzero and reorganise the effective coefficients.
\end{enumerate}
Therefore, in the commuting sector, disagreement between $K_2$ and exact ED is interpreted as missing cumulant renormalisation, not a breakdown of the formalism.

\subsection{Level-resolved diagnostics}

For diagonal $H_S=\mathrm{diag}(E_i)$ and diagonal populations $p_i$,
\begin{equation}
\Delta_i
:=
E_i-E_0+\beta^{-1}\ln\frac{p_i}{p_0}
=
\sum_{n\ge 1}\alpha_{2n}\left(f_i^{2n}-f_0^{2n}\right),
\label{eq:v2_level_shift}
\end{equation}
which we use as a direct numerical diagnostic of cumulant truncation quality.
