\section{\label{sec:consequences}Consequences in the Commuting Sector}

The commuting theorem gives a constructive answer to the HMF question in this regime:
all bath effects are encoded in scalar coefficients multiplying powers of the conserved coupling operator.

\subsection{Gaussian baseline as first hierarchy level}

For Gaussian baths only $\alpha_2$ survives, and
\(
H_{\mathrm{MF}}=H_S-\alpha_2 f^2+\text{const}.
\)
This reproduces the standard reorganisation-energy picture while keeping an exact finite-bath reference.

\subsection{Non-Gaussian corrections as renormalisation hierarchy}

For bounded or finite baths (e.g. transverse-spin baths), higher cumulants do not vanish identically. The sequence
\(
\alpha_4,\alpha_6,\dots
\)
adds higher powers of $f^2$ and shifts sector weights in a controlled way. The numerical ladder
\(
K_2 \rightarrow K_2{+}K_4 \rightarrow K_2{+}K_4{+}K_6
\)
should therefore be read as a renormalisation sequence: each level refines an already well-defined effective Hamiltonian.

\subsection{Objection handling}

The commuting data resolve three standard objections:
\begin{enumerate}
\item \textit{``Quartic terms are ad hoc''}: finite-spin baths generate them directly from $Z_B(\theta)$ derivatives.
\item \textit{``Higher orders invalidate low-order structure''}: instead they correct it systematically, with measurable error reduction.
\item \textit{``Agreement is fitting artefact''}: predictions are parameter-free once $(H_B,B)$ are fixed, and are benchmarked against explicit ED trace-outs.
\end{enumerate}
