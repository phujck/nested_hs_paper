\section{\label{sec:consequences}Consequences: Algebraic Closure and the Operator Content of the HMF}

The commuting-sector theorem, Eq.~\eqref{eq:HMF_commuting_full}, gives the Hamiltonian of mean force as a power series in the coupling operator:
\begin{equation}
    H_{\mathrm{MF}} = H_S - \frac{1}{\beta}\sum_{n=1}^{\infty}\alpha_{2n}\,f^{2n}.
    \label{eq:HMF_polynomial_recap}
\end{equation}
This is a polynomial in a single operator, determined entirely by the bath cumulants. But the physical content of this polynomial depends crucially on the \emph{algebra} generated by even powers of $f$: how many linearly independent operators appear in the sequence $f^2, f^4, f^6, \ldots\,$? If this algebra closes at finite dimension, then the bath can only renormalise a finite set of operator structures---regardless of how exotic the bath statistics. This is the central structural insight, and it has far-reaching consequences.

\subsection{Algebraic closure and renormalisation}

Define the \emph{even-coupling algebra} as the unital algebra generated by $f^2$:
\begin{equation}
    \mathcal{A}_f = \mathrm{span}\{I,\,f^2,\,f^4,\,f^6,\ldots\}.
    \label{eq:algebra_def}
\end{equation}
By the Cayley--Hamilton theorem, any operator on a $d$-dimensional Hilbert space satisfies a polynomial relation of degree $d$, so the powers $f^2, f^4,\ldots$ eventually become linearly dependent. The algebra $\mathcal{A}_f$ therefore closes at dimension $\dim\mathcal{A}_f \leq d$.

Closure has a sharp physical consequence: \emph{once the algebra closes, the bath cannot generate any new operator structures---it can only renormalise the coefficients of the existing ones}. In the language of the HMF:
\begin{equation}
    H_{\mathrm{MF}} = H_S + \sum_{j=0}^{D-1} c_j(\alpha_{2n},\beta)\,f^{2j},
    \label{eq:HMF_closed_form}
\end{equation}
where $D = \dim\mathcal{A}_f$ and the $c_j$ absorb all contributions from the infinite cumulant series. The environment is completely characterised, from the system's point of view, by $D-1$ real numbers (the coefficients $c_j$ for $j\geq 1$).

This is worth emphasising: a $d$-level system coupled to \emph{any} bath through \emph{any} single coupling operator in the commuting sector has an HMF that is determined by at most $d-1$ parameters. An infinite number of bath cumulants collapse into a finite-dimensional operator deformation.

\subsection{Qubits: automatic triviality}

For a two-level system, any traceless operator $f$ satisfies $f^2 = -(\det f)\,I \propto I$. The algebra closes at $D=1$ (with $f^0 = I$): there are no independent even powers beyond the identity. Consequently, every even power $f^{2n} \propto I$, and the infinite sum in Eq.~\eqref{eq:HMF_polynomial_recap} collapses to a scalar energy shift:
\begin{equation}
    H_{\mathrm{MF}} = H_S + c_0\,I.
    \label{eq:HMF_qubit}
\end{equation}
No matter how exotic the bath statistics, a qubit coupled through a single operator in the commuting sector experiences \emph{no operator deformation at all}---only a global energy shift. The non-Gaussian hierarchy is algebraically invisible, not because the cumulants vanish, but because $f^2$ has already closed the algebra. A qubit coupled to a wildly non-Gaussian spin bath has the same \emph{form} of HMF as one coupled to a harmonic oscillator.

\subsection{Qutrits and higher: non-trivial deformation}

The situation changes qualitatively at $d\geq 3$. For the qutrit clock operator $f = Z_3 = \mathrm{diag}(1,\omega,\omega^2)$ with $\omega = e^{2\pi i/3}$, the minimal polynomial is $f^3 - I = 0$, so $f^2 = f^\dagger$ is independent of $I$ and $f$, but $f^4 = f$ closes the algebra. The HMF becomes
\begin{equation}
    H_{\mathrm{MF}} = H_S + c_0\,I + c_1\,f^2,
    \label{eq:HMF_qutrit}
\end{equation}
where $c_1$ absorbs contributions from the fourth and all higher cumulants. The fourth cumulant generates a genuinely new operator, $f^2 = f^\dagger$, that is not proportional to $f$ or $I$. This is the first case where non-Gaussianity produces qualitatively new physics: the bath \emph{deforms} the effective Hamiltonian, not just its eigenvalues.

More generally, for a $d$-level system with $f$ having $d$ distinct eigenvalues, each cumulant order up to $2\lfloor d/2\rfloor$ contributes an algebraically independent correction before closure forces all higher orders to be linearly dependent.

\subsection{Conditions for a local HMF}

The algebraic structure has deeper implications when we ask: \emph{when can the bath influence be absorbed into a Hamiltonian of mean force at all?} In the general (non-commuting) case, the influence functional produces a nonlocal-in-time self-interaction that cannot be reduced to an operator on the system Hilbert space alone. The HMF exists only when this reduction is possible.

In the commuting sector, the reduction is always possible---the theorem guarantees it. But the result lives in the algebra $\mathcal{A}_f$. If $\mathcal{A}_f$ is a proper subalgebra of the full operator algebra on $\mathcal{H}_S$, then the bath couples to a restricted set of degrees of freedom, and the ``environment'' defined by the HS fields can be thought of as acting on the subalgebra alone.

This perspective connects naturally to the reaction-coordinate formalism~\cite{strathearn2018efficient,nazir2018reaction}, where one absorbs part of the bath into a renormalised system. In our framework, the algebraic closure of $f^{2n}$ determines precisely \emph{which} operators the bath can dress. If the algebra is small (as for qubits), the bath has nowhere to go: it can only shift global scales. If the algebra is large (as for many-body systems with a high-dimensional coupling), new effective interactions emerge at each cumulant order.

The analogy with reaction coordinates is deep: in both cases, part of the environment's influence is absorbed into a redefined system Hamiltonian, with the ``reaction coordinate'' identified by the algebraic structure of the coupling. The difference is that here, the couplings algebra determines the \emph{maximal possible deformation} of the HMF, irrespective of the bath model---it is a property of the system and the coupling alone.

\subsection{Integrable chains: bath as a function of a conserved charge}

The polynomial structure becomes physically transparent---and remarkably powerful---when applied to many-body systems. Consider an integrable spin chain, such as the XXZ model:
\begin{equation}
    H_S = \sum_i \big[J(X_iX_{i+1} + Y_iY_{i+1}) + \Delta\,Z_iZ_{i+1}\big],
    \label{eq:XXZ}
\end{equation}
which conserves total magnetisation $S_z = \sum_i Z_i$. If the bath couples to this conserved quantity, $f = S_z$, then $[H_S, f] = 0$ and the HMF takes the form
\begin{equation}
    H_{\mathrm{MF}} = H_S + F(S_z),\quad F(m) = \sum_{n\geq 1}\alpha_{2n}\,m^{2n}.
    \label{eq:HMF_chain}
\end{equation}

The operator $(S_z)^{2n}$ is a sum of $n$-body diagonal interactions. Expanding:
\begin{align}
    (S_z)^2 &= N\,I + 2\!\sum_{i<j}Z_iZ_j, \label{eq:Sz2}\\[4pt]
    (S_z)^4 &\supset \text{1-, 2-, 3-, and 4-body $Z$ strings}.
    \label{eq:Sz4}
\end{align}
A Gaussian bath ($\alpha_2$ only) generates infinite-range two-body Ising interactions $Z_iZ_j$ across the entire chain. A non-Gaussian bath adds three-body, four-body, and higher diagonal interactions.

But here is the key simplification: because $S_z$ is a conserved charge, the algebra of even polynomials $\mathcal{A}_{S_z}$ acts within each magnetisation sector as a scalar function of $m$. The $2^N$-dimensional operator algebra of the chain reduces to an $N+1$-dimensional function space (one coefficient per sector). No matter how many-body the formal operator expansion looks, the bath influence is captured by a single nonlinear reshaping of the energy landscape:
\begin{equation}
    F(S_z)\big|_m = \sum_{n\geq 1}\alpha_{2n}\,m^{2n}.
    \label{eq:F_sectors}
\end{equation}

This has a satisfying structural consequence: \emph{the bath deformation preserves integrability}. Since $F(S_z)$ commutes with $H_S$, the system $H_S + F(S_z)$ has the same eigenstates as $H_S$---only the energies are shifted, differently in each magnetisation sector. The integrable structure (Bethe ansatz solvability, conservation laws) survives intact. Non-Gaussianity can reshape the spectrum but cannot break integrability.

To summarise: the even-coupling algebra $\mathcal{A}_f$ is the master object. It determines the dimension of the space in which the bath can act, the conditions under which a local HMF exists, and the maximal operator content of the bath-induced deformation. For qubits, the algebra is trivial and the bath is invisible. For many-body systems coupled through a conserved charge, the algebra is large but block-diagonal, and the bath reduces to a nonlinear function of that charge.
