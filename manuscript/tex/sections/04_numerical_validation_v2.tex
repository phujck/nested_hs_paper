\section{\label{sec:numerics}Numerical Validation (Commuting Sector)}

All V2 simulations are produced by \texttt{run\_nested\_hs\_suite\_v2.py},
with figures from \texttt{plot\_nested\_hs\_suite\_v2.py} and claim checks from
\texttt{validate\_nested\_hs\_claims\_v2.py} in \path{simulations/src/}.
To preserve continuity with V1, we retain the multi-qubit magnetisation and qutrit-clock phase benchmarks before the finite-bath ED qutrit tests.

For all ED benchmarks we use
\begin{equation}
\rho_S^{\mathrm{ED}}(\beta)=\frac{\Tr_B\,e^{-\beta H_{\mathrm{tot}}}}{\Tr\,e^{-\beta H_{\mathrm{tot}}}},
\label{eq:v2_rho_ed}
\end{equation}
and define the exact mean-force operator
\begin{equation}
H_{\mathrm{MF}}^{\mathrm{ED}}=-\beta^{-1}\log \rho_S^{\mathrm{ED}}+\mathrm{const}.
\end{equation}
By construction, rebuilding $\rho$ from $H_{\mathrm{MF}}^{\mathrm{ED}}$ is exact up to numerical precision:
\(
D_{\mathrm{HMF}}=\tfrac12\|\rho_S^{\mathrm{ED}}-\mathcal N[e^{-\beta H_{\mathrm{MF}}^{\mathrm{ED}}}]\|_1\approx 0.
\)

\subsection{Multi-qubit magnetisation benchmark (NHS-CNG-MAG-v2)}

\paragraph{Model.}
This is the V1 multi-qubit commuting benchmark recast in the V2 suite.
We consider an $N$-qubit model resolved by total magnetisation
\(
S_z=\sum_{i=1}^N \sigma_i^z
\),
with sector probabilities
\begin{equation}
p(m)\propto g(m)\,\exp\!\left(k_2 m^2 + k_4 m^4\right),
\end{equation}
with magnetisation sectors $m\in\{-N,-N+2,\dots,N\}$ and binomial degeneracy $g(m)$.
For the present test, $(k_2,k_4)$ are treated as controlled effective cumulant coefficients so that the exact sector law is known.

\paragraph{Implementation.}
We compare exact analytic sector probabilities against HS Monte Carlo in two branches:
\begin{enumerate}
\item PSD branch: $k_4>0$;
\item non-PSD branch: $k_4<0$ with complex contour rotation in the quartic auxiliary field.
\end{enumerate}
Diagnostics include trace distance, max absolute error, and imaginary leakage
\(
\sum_m |\Im w_m| \big/ \sum_m |\Re w_m|.
\)

\paragraph{Result statement.}
Figure~\ref{fig:nhs_synth_nonpsd_v2}(A,B) shows analytic-vs-MC overlays in both PSD and non-PSD branches.
Figure~\ref{fig:nhs_synth_nonpsd_v2}(C) shows convergence with sample count.
This reinstates the multi-qubit magnetisation stress test while validating contour-rotated HS sampling before the finite-bath qutrit benchmarks.

\begin{figure*}[t]
    \centering
    \IfFileExists{../../simulations/results_v2/figures/nhs_synth_nonpsd_v2.pdf}{
    \includegraphics[width=\textwidth]{../../simulations/results_v2/figures/nhs_synth_nonpsd_v2.pdf}
    }{
    \fbox{\parbox{0.95\textwidth}{\centering
    Placeholder: run \texttt{plot\_nested\_hs\_suite\_v2.py} to generate\\
    \texttt{simulations/results\_v2/figures/nhs\_synth\_nonpsd\_v2.pdf}.}}
    }
    \caption{\label{fig:nhs_synth_nonpsd_v2}
    \textbf{NHS-CNG-MAG-v2 (multi-qubit magnetisation sectors).}
    Commuting $S_z$-sector benchmark with quartic cumulants.
    (A) PSD branch ($k_4>0$): analytic vs Monte Carlo sector probabilities.
    (B) non-PSD branch ($k_4<0$): contour-rotated Monte Carlo vs analytic.
    (C) Trace-distance convergence with sample count $M$.
    }
\end{figure*}

\subsection{Qutrit clock phase benchmark (NHS-CNG-PHASE-v2)}

\paragraph{Model.}
We use the commuting qutrit-clock setting with complex phase in the quartic cumulant:
\begin{equation}
\kappa_4=r\,e^{i\phi},\qquad \kappa_2=\kappa_4^\ast,
\end{equation}
and populations sampled versus $\phi/\pi$ for several correlation lengths $\tau_c$.

\paragraph{Implementation.}
For each $\phi$ and $\tau_c$, we compare analytic phase-dependent populations to discretised functional HS Monte Carlo.
To reduce Monte Carlo jaggedness, each $(\phi,\tau_c)$ point is estimated from replica-averaged sampling with fixed total sample budget.
This benchmark isolates phase handling and complex auxiliary-field sampling quality.

\paragraph{Result statement.}
Figure~\ref{fig:nhs_phase_clock_v2}(A) shows smooth analytic phase curves with Monte Carlo overlays (representative $\tau_c$) for direct visual matching.
Figure~\ref{fig:nhs_phase_clock_v2}(B) shows trace-distance error versus phase for each $\tau_c$ with replica uncertainty bands, confirming stable phase tracking.

\begin{figure*}[t]
    \centering
    \IfFileExists{../../simulations/results_v2/figures/nhs_cng_phase_clock_v2.pdf}{
    \includegraphics[width=\textwidth]{../../simulations/results_v2/figures/nhs_cng_phase_clock_v2.pdf}
    }{
    \fbox{\parbox{0.95\textwidth}{\centering
    Placeholder: run \texttt{plot\_nested\_hs\_suite\_v2.py} to generate\\
    \texttt{simulations/results\_v2/figures/nhs\_cng\_phase\_clock\_v2.pdf}.}}
    }
    \caption{\label{fig:nhs_phase_clock_v2}
    \textbf{NHS-CNG-PHASE-v2.}
    Qutrit clock phase benchmark.
    (A) Phase-resolved populations: smooth analytic curves with Monte Carlo overlays.
    (B) Trace-distance error $D(\mathbf p^{\mathrm{MC}},\mathbf p^{\mathrm{analytic}})$ vs phase, shown for each $\tau_c$ with replica uncertainty bands.
    }
\end{figure*}

\subsection{Commuting Gaussian qutrit benchmark (NHS-CG-1-v2)}

\paragraph{Model.}
We use
\begin{equation}
H_S=\mathrm{diag}(0,0.9,1.8),\qquad f=\mathrm{diag}(0,1,2),
\end{equation}
coupled to a finite-mode harmonic bath:
\begin{equation}
H_B=\sum_k\omega_k(a_k^\dagger a_k+\tfrac12),\qquad
H_I=f\otimes\sum_k c_k x_k,
\end{equation}
with $x_k=(a_k+a_k^\dagger)/\sqrt{2\omega_k}$ and Ohmic discretisation.

\paragraph{Implementation.}
At each $(\beta,g)$ we compare:
\begin{enumerate}
\item finite-bath ED;
\item commuting analytic prediction
\(
\rho_S^{\mathrm{an}}\propto e^{-\beta(H_S-\lambda_{\mathrm{disc}}f^2)}
\),
\(
\lambda_{\mathrm{disc}}=\sum_k c_k^2/(2\omega_k^2)
\);
\item scalar and path HS estimators.
\end{enumerate}
We also report $D_{\mathrm{HMF}}$ from exact HMF reconstruction.

\paragraph{Observable definitions.}
Primary observables are $p_2=\langle 2|\rho_S|2\rangle$ and trace distances
\(
D(\rho_a,\rho_b)=\tfrac12\|\rho_a-\rho_b\|_1
\).
We additionally monitor
\begin{equation}
\lambda_{\mathrm{est}}=
\frac{E_2-E_0+\beta^{-1}\ln(p_2/p_0)}{f_2^2-f_0^2},
\label{eq:v2_lambda_est}
\end{equation}
for $\beta$-invariance and $g^2$ scaling.

\paragraph{Result statement.}
Figure~\ref{fig:nhs_cg_v2}(A) shows ED/analytic/HS agreement in the coupled population.
Figure~\ref{fig:nhs_cg_v2}(B) shows cutoff convergence and Monte Carlo scaling.
The exact-HMF reconstruction diagnostic is numerically at machine precision across the grid.

\begin{figure*}[t]
    \centering
    \IfFileExists{../../simulations/results_v2/figures/nhs_cg_1_v2.pdf}{
    \includegraphics[width=\textwidth]{../../simulations/results_v2/figures/nhs_cg_1_v2.pdf}
    }{
    \fbox{\parbox{0.95\textwidth}{\centering
    Placeholder: run \texttt{plot\_nested\_hs\_suite\_v2.py} to generate\\
    \texttt{simulations/results\_v2/figures/nhs\_cg\_1\_v2.pdf}.}}
    }
    \caption{\label{fig:nhs_cg_v2}
    \textbf{NHS-CG-1-v2 (commuting Gaussian qutrit).}
    (A) Coupled-level population $p_2$ vs inverse temperature: ED, analytic prediction, and HS estimators.
    (B) ED-to-analytic cutoff convergence with HS residual baselines and Monte Carlo scaling inset.
    }
\end{figure*}

\subsection{Commuting non-Gaussian finite-spin qutrit benchmark (NHS-CNG-1-v2)}

\paragraph{Model.}
We keep the same commuting qutrit and use a finite transverse-spin bath:
\begin{equation}
H_B=\sum_{j=1}^{N_B}\omega_j\sigma_j^x,\qquad
B=\sum_{j=1}^{N_B} c_j \sigma_j^z,\qquad
H_I=g\,f\otimes B.
\end{equation}

\paragraph{Implementation.}
For each $(\beta,g)$:
\begin{enumerate}
\item compute $\rho_S^{\mathrm{ED}}$ from exact diagonalisation;
\item extract $\alpha_2,\alpha_4,\alpha_6$ from derivatives of $\log Z_B(\theta)$;
\item compare $K_2$, $K_2{+}K_4$, and stability-gated $K_2{+}K_4{+}K_6$ truncations.
\end{enumerate}
The sixth-order term is included only when the derivative-stability diagnostic satisfies
\(
\epsilon_6^{\mathrm{stab}}\le 0.35
\);
otherwise $\alpha_6$ is clipped to zero.

\paragraph{Observable definitions.}
Primary residuals:
\(
D_{(2)}, D_{(2+4)}, D_{(2+4+6)}.
\)
Renormalisation indicators:
\(
\chi_4=\alpha_4/\alpha_2^2,\;
\chi_6=\alpha_6/|\alpha_2|^3.
\)
We also monitor exact-HMF reconstruction via $D_{\mathrm{HMF}}$.

\paragraph{Result statement.}
Figure~\ref{fig:nhs_cng_v2}(A) gives ED-vs-truncation population overlays.
Figure~\ref{fig:nhs_cng_v2}(B) shows truncation error versus coupling at low/mid/high $\beta$ slices.
Figure~\ref{fig:nhs_cng_v2}(C) gives a full validity map, $\log_{10}(D_{K2}/D_{K2+K4})$, identifying where quartic truncation helps or breaks down.
Figure~\ref{fig:nhs_cng_v2}(D) shows rescaled cumulant coefficients $\alpha_2/g^2$ and $\alpha_4/g^4$, clarifying that the quartic coefficient is not negligible once coupling scaling is factored out.
The exact-HMF reconstruction remains numerically exact; residuals in (A-D) are truncation error, not an ED/HMF inconsistency.

\begin{figure*}[t]
    \centering
    \IfFileExists{../../simulations/results_v2/figures/nhs_cng_1_v2.pdf}{
    \includegraphics[width=\textwidth]{../../simulations/results_v2/figures/nhs_cng_1_v2.pdf}
    }{
    \fbox{\parbox{0.95\textwidth}{\centering
    Placeholder: run \texttt{plot\_nested\_hs\_suite\_v2.py} to generate\\
    \texttt{simulations/results\_v2/figures/nhs\_cng\_1\_v2.pdf}.}}
    }
    \caption{\label{fig:nhs_cng_v2}
    \textbf{NHS-CNG-1-v2 (commuting non-Gaussian finite-spin qutrit).}
    (A) Qutrit population overlays at representative $(\beta,g)$ for ED and truncation ladder.
    (B) Trace-distance vs coupling for low/mid/high $\beta$ slices, comparing $K_2$ and $K_2{+}K_4$.
    (C) Validity map $\log_{10}(D_{K2}/D_{K2+K4})$ in the $(\beta,g)$ plane; positive values indicate quartic improvement.
    (D) Rescaled cumulants $\alpha_2/g^2$ and $\alpha_4/g^4$ with coupling-collapse bands.
    }
\end{figure*}

Non-commuting validations are intentionally deferred to the \textit{what\_rules\_equilibrium} paper; this manuscript version is strictly commuting-sector.
