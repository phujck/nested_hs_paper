\section{\label{sec:discussion}Discussion}

We have shown that the imaginary-time influence functional of a general (non-Gaussian) bath admits an exact, constructive representation as a nested hierarchy of Gaussian auxiliary fields. The construction requires only that the cumulant series exists; it places no conditions on the sign, definiteness, or boundedness of the cumulant kernels. Contour rotation of the auxiliary fields---eigenmode by eigenmode---suffices to define the Gaussian integrals for arbitrary complex or indefinite kernels.

In the commuting sector $[H_S,f]=0$, the entire hierarchy collapses analytically. The Hamiltonian of mean force takes the form of a renormalised polynomial in the coupling operator, with coefficients determined by the integrated cumulants of the bath. This is exact, non-perturbative, and---as the numerical demonstrations confirm---quantitatively reliable even in parameter regimes where the cumulant kernels are strongly non-PSD.

The consequences differ sharply by system dimension. For qubits, the algebraic closure of $f$ renders the non-Gaussian hierarchy invisible: every cumulant contributes to the same two independent operator coefficients. For qutrits and higher-dimensional systems, new operator content enters at each cumulant order, up to the dimension of the coupling algebra. For many-body systems coupled through a conserved charge, the bath generates an arbitrary even function of that charge---reshaping the energy landscape across symmetry sectors without breaking integrability.

Several directions suggest themselves. The nested HS representation extends formally to the non-commuting sector $[H_S,f]\neq 0$, where the auxiliary fields no longer average analytically but can be sampled numerically. This would provide a stochastic formulation of the full non-Gaussian influence functional, complementing existing hierarchical methods~\cite{tanimuraReducedHierarchicalEquations2014}. The construction could also be applied to real-time dynamics, where contour issues become more delicate but the algebraic structure is identical. Finally, the observation that non-Gaussianity preserves integrability when acting through a conserved charge may have implications for quantum thermodynamics in integrable systems---an area where the interplay between conservation laws and bath statistics is not yet well understood.
